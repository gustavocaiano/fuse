% Chapter 2

\chapter{Estudo do Estado da Arte}
\label{chap:estado_da_arte}

Este capítulo apresenta a revisão de literatura fundamentada no âmbito do projeto, analisando o estado atual das tecnologias críticas para o desenvolvimento do \gls{FUSE}. O objetivo principal desta revisão é responder às sub-questões operacionais previamente identificadas, nomeadamente a RQ1.1 e a RQ1.2, estabelecendo uma base teórica sólida para as decisões de arquitetura e segurança.

Adicionalmente, e dada a natureza aplicada da RQ1.3, é conduzido um estudo técnico aprofundado sobre os paradigmas atuais de visão computacional. Este estudo visa não apenas identificar o estado da arte, mas também comparar padrões, pipelines de processamento e modelos de \gls{AI} passíveis de integração na plataforma.

\section{Processo de Investigação}
\label{sec:processo_investigacao}

O Processo de investigação foi guiado pelo mapeamento das duas primeiras sub-questões da presente dissertação em questões de pesquisa com foco na revisão literária, conforme descrito na Tabela \ref{tab:questoes_pesquisa}.

\begin{table}[H]
\centering
\caption{Questões de Pesquisa Orientadas à Revisão Literária}
\label{tab:questoes_pesquisa}
\begin{tabular}{llp{5cm}p{5cm}}
\toprule
\textbf{RQ ID} & \textbf{LRRQ ID} & \textbf{Description} & \textbf{Tópicos e Keywords de pesquisa} \\
\midrule
RQ1.1 & LRRQ1 & Quais são as arquiteturas de referência e padrões de design de software descritos na literatura para a abstração e interoperabilidade de dispositivos \gls{IoT} heterogéneos? & \gls{IoT} Interoperability, Hardware Abstraction Layer, Middleware patterns, \gls{ONVIF} standardization, Heterogeneous device integration. \\
RQ1.2 & LRRQ2 & Qual o estado da arte em protocolos de comunicação segura e \glspl{VPN} para acesso remoto e \textit{streaming}? & Secure Tunneling Protocols, \gls{VPN} performance analysis, Streaming encryption, \gls{NAT} Traversal, \gls{ZTNA}. \\
\bottomrule
\end{tabular}
\end{table}

Cada questão de pesquisa será enquadrada de acordo com o modelo \gls{PICOCS} \cite{Alberto2015}, garantindo o rigor na seleção das fontes utilizadas. A informação será depois selecionada seguindo o fluxo \gls{PRISMA} \cite{Moher2010}. O processo inclui a definição de keywords de pesquisa, a aplicação estrita de critérios de inclusão e exclusão, seguida de uma filtragem em etapas e, finalmente, uma discussão crítica sobre a aplicabilidade dos estudos selecionados para a arquitetura do \gls{FUSE}.

Como fator inclusivo de seleção, selecionou-se, por exemplo, a data de publicação posterior a 2018. A escolha deste intervalo visa garantir que as arquiteturas, frameworks e algoritmos analisados representam o atual estado da arte, evitando a adoção de paradigmas que, embora válidos no passado, não refletem as necessidades de desempenho e escalabilidade dos sistemas modernos. Foram privilegiados artigos revistos por pares, normas técnicas e literatura técnica amplamente reconhecida. Excluíram-se estudos puramente teóricos sem aplicação prática ao domínio da videovigilância ou da integração de dispositivos, tal como trabalhos relativos a soluções proprietárias fechadas sem documentação técnica pública suficiente.


\section{LRRQ1 - Arquiteturas de Abstração e Interoperabilidade IoT}
\label{sec:lrrq1_iot}

Nesta secção, a literatura é revista de forma a responder à questão de investigação "Quais são as arquiteturas de referência e padrões de design de software descritos na literatura para a abstração e interoperabilidade de dispositivos IoT heterogéneos?".

\subsection{Processo de Pesquisa}

A questão de investigação foi estruturada de acordo com o modelo PICOCS apresentado na Tabela \ref{tab:lrrq1_picocs}.

\begin{table}[H]
	\centering
	\caption{Modelo PICOCS para a LRRQ1}
	\label{tab:lrrq1_picocs}
	\begin{tabular}{p{0.15\linewidth}p{0.35\linewidth}p{0.2\linewidth}p{0.2\linewidth}}
		\toprule
		\textbf{PICOCS} & \textbf{Parte da RQ} & \textbf{I/E} & \textbf{Sub string} \\
		\midrule
		\textbf{P}  & Estudos envolvendo dispositivos IoT heterogéneos e CCTV & I - Estudos que considerem heterogeneidade \newline E - Anteriores a 2018 & >=2018; IoT; Heterogeneous; CCTV \\
		\textbf{I}  & Arquiteturas de software, Middleware e Camadas de Abstração & I - Estudos sobre padrões de design e abstração \newline E - Soluções proprietárias fechadas & Software Architecture; Middleware; Abstraction Layer; Design Patterns \\
		\textbf{C}  & --- & --- & --- \\
		\textbf{O} & Arquiteturas de referência e interoperabilidade & I - Estudos que propõem ou validam arquiteturas & Architecture; Interoperability; Framework \\
		\textbf{C} & Engenharia de Software e Sistemas Distribuídos & I - Académico \newline I - Indústria & Software Engineering; Distributed Systems \\
		\textbf{S}  & Estudos de caso e Propostas de Arquitetura & I - Case Studies \newline I - System Proposals \newline E - Opiniões sem validação & Case study; Proposal; Prototype \\
		\bottomrule
	\end{tabular}
\end{table}

A Tabela \ref{tab:lrrq1_picocs} inclui os critérios de inclusão e exclusão (I/E) relacionados com cada parte da questão de investigação, bem como as possíveis \textit{sub-strings} utilizadas para conduzir a pesquisa nas bases de dados científicas. Apenas estudos de 2018 em diante que contemplem arquiteturas de software ou middleware para integração de dispositivos heterogéneos foram considerados.

A pesquisa foi conduzida em três bases de dados principais: IEEE Xplore (282 registos), ACM Digital Library (94 registos) e ScienceDirect (59 registos), totalizando 435 registos identificados através de pesquisa sistemática. Adicionalmente, foi identificado 1 registo \cite{resende_sistema_2025} através de outras fontes, resultando num total de 436 registos na fase de identificação. 

Após a remoção de duplicados (1 registo), foram analisados 435 registos na fase de \textit{screening} através da leitura de títulos e resumos. Desta análise, 412 registos foram excluídos por não cumprirem os critérios de inclusão, nomeadamente por focarem-se puramente em algoritmos de visão computacional sem arquitetura de sistema, abordarem domínios como agricultura ou saúde sem componente de vídeo, ou simplemsmente não se adequar ao tema de integração de dispositivos \gls{IoT} ou de camadas de abstração em design de software. Os 23 registos restantes foram avaliados em texto completo, dos quais 12 foram excluídos por falta de detalhes de implementação, ausência de diagrama de arquitetura claro, ou por serem propostas teóricas sem validação prática. O processo resultou na inclusão final de 11 estudos na revisão qualitativa, conforme ilustrado na Figura \ref{fig:lrrq1_prisma}.

\begin{figure}[H]
\centering
\includegraphics[width=\textwidth]{ch2/assets/LRRQ1-prisma.png}
\caption{Fluxo PRISMA para a LRRQ1}
\label{fig:lrrq1_prisma}
\end{figure}

O rigor na filtragem aplicada, reduzindo de 435 para 11 estudos incluídos, garantiu que apenas trabalhos com arquiteturas de sistema relevantes, detalhadas e validadas fossem analisados, assegurando a qualidade e relevância da revisão de literatura para fundamentar as decisões arquiteturais do \gls{FUSE}.

\subsection{Discussão}

A análise dos 11 estudos incluídos revela uma convergência clara na literatura: a integração direta de dispositivos \gls{IoT} heterogéneos sem uma camada intermediária resulta em sistemas monolíticos e de difícil manutenção.A diversidade de protocolos (\gls{RTSP}, \gls{HTTP}, \gls{MQTT}), algoritmos de codificação de vídeo (H.264, H.265, MJPEG) e especificidades de fabricantes exige, invariavelmente a implementação de uma camada de abstração que atue como tradutor universal. Como observado no estudo \cite{mesmoudi_design_2018}, a ausência desta camada força o sistema central a lidar com a complexidade de hardware, violando princípios de desacoplamento. A literatura valida que a solução reside numa arquitetura que isole a lógica de negócio das particularidades dos dispositivos.

\subsubsection{Padrões Arquiteturais: De Gateways a Microserviços Agnósticos}

A evolução dos padrões arquiteturais na literatura aponta para um distanciamento de \textit{gateways }simples em direção a microserviços inteligentes. Inicialmente, soluções como o IoTM2B \cite{barros_iot_2019} e Smart Gateway \cite{mesmoudi_design_2018} validaram o uso de um ponto central para tradução de protocolos (ex: converter \gls{MQTT}/\gls{CoAP} para \gls{HTTP}). No entanto, o trabalho de \cite{barros_iot_2019} identifica uma limitação crítica: a normalização para \gls{HTTP} POST, embora eficaz para envio de leituras de um sensor, introduz um gargalo de desempenho inaceitável para \textit{streaming} de vídeo contínuo.

A resposta a esta limitação encontra-se na adoção de microserviços com o padrão \textit{Mediator}, como proposto no estudo \cite{dobaj_microservice_2018}. Esta abordagem introduz uma Camada de Abstração de Dados dedicada que desacopla a lógica de serviço da comunicação com o dispositivo. Mais relevante para o conceito do \gls{FUSE} é a introdução do padrão ''Access Mapper'' ou ''Digital Twin'' pelo trabalho \cite{mafamane_study_2021}. Neste modelo, a aplicação central interage exclusivamente com um ''Dispositivo Virtual'' padronizado, tornando-se completamente agnóstica ao hardware físico. Este é o fundamento teórico para um componente da solução a desenvolver, ou seja, uma camada que encapsula a complexidade do driver proprietário e expõe uma interface limpa e uniforme.

\subsubsection{Abstração de Media e Normalização de Streaming}

A abstração em sistemas de videovigilância exige também a normalização do próprio \textit{streaming} de vídeo. A conexão direta entre uma aplicação web e câmaras de vigilância apresenta desafios técnicos significativos devido à natureza dos fluxos de vídeo brutos e algoritmos de codificação de vídeo utilizados. A literatura, nomeadamente o estudo \cite{schwarzer_ial_2021}, suporta a implementação de uma camada intermédia responsável pelo tratamento do \textit{input} e gestão desto \textit{streaming}, expondo-os posteriormente num formato padronizado e otimizado para consumo web (como \gls{HLS} ou \gls{WebRTC}).

Esta abordagem desacopla a aplicação cliente da complexidade de conexão direta às câmaras: o cliente consome sempre um fluxo normalizado a partir da camada de abstração. Embora em \cite{schwarzer_ial_2021} e \cite{barra_exploring_2024} se discuta também o processamento semântico destes fluxos em Edge (análise de vídeo), este tópico específico de automação e visão computacional será aprofundado na Secção \ref{sec:estado_arte_visao}. Para a LRRQ1, o contributo essencial é a arquitetura que garante a estabilidade e abstração do \textit{streaming}.

\subsubsection{Escalabilidade e Adaptabilidade à Rede}

A viabilidade futura do sistema depende da sua capacidade de escalar para novos dispositivos e adaptar-se a redes instáveis. O estudo \cite{mesmoudi_design_2018} demonstra a importância da descoberta automática de dispositivos, tecnicamente designada por \textit{service discovery}, para detetar novas câmaras sem configuração manual, um requisito essencial para a usabilidade. No que toca à rede, o artigo \cite{george_distributed_2019} introduz o conceito vital de ``Latency-Accuracy Trade-off'', onde o middleware ajusta dinamicamente parâmetros (como a resolução do vídeo) face à congestão da rede. A validação empírica desta topologia distribuída é fornecida pelo estudo \cite{gomez_strategies_2023}, que comprova uma redução de latência de até \percentage{98} em arquiteturas "Edge Computing" comparativamente à Cloud pura para serviços críticos. Adicionalmente, o trabalho \cite{immich_multi-tier_2019} demonstra as vantagens de uma arquitetura baseada em cadeias de módulos independentes. Esta modularidade permite que componentes distintos (como a o processamento e gestão do \textit{streaming}, a transcodificação e a análise do mesmo) operem de forma independente e encadeada. Para o \gls{FUSE}, isto significa que é possível atualizar ou substituir módulos individuais sem comprometer a estabilidade do sistema base.

\subsubsection{Análise Crítica}

Apesar da diversidade de soluções, a revisão identifica lacunas relevantes para o contexto do \gls{FUSE}. Primeiro, a maioria das arquiteturas foca-se em sensores e transmissão de leituras dos mesmos, negligenciando os requisitos específicos de largura de banda do vídeo \cite{barros_iot_2019}. Adicionalmente, embora o estudo \cite{gomez_strategies_2023} comprove a baixa latência em Edge, a sua dependência de hardware compatível com \gls{TSN} torna a solução impraticável para ambientes domésticos com equipamentos de rede convencionais, exigindo abordagens de software mais flexíveis.

Segundo, embora existam padrões sintáticos, o estudo \cite{muniz_pragmatic_2019} alerta para a falta de ``Interoperabilidade Pragmática'', ou seja, a capacidade dos dispositivos comunicarem a intenção do dado (ex: urgência de um alerta) e não apenas o conteúdo. O trabalho de \cite{resende_sistema_2025} valida a viabilidade técnica de usar plataformas open-source como o ZoneMinder em hardware modesto (\textit{Raspberry Pi}), mas a ausência dessa camada semântica e pragmática limita a capacidade do sistema reagir contextualmente e expandir o leque de câmaras suportadas de forma escalável. Isto reforça a necessidade da arquitetura distribuída, eficiente e modular proposta para o \gls{FUSE}.

Esta camada tem como responsabilidades críticas: (1) normalizar a comunicação com dispositivos heterogéneos para uma interface agnóstica \cite{barros_iot_2019,mesmoudi_design_2018}, (2) abstrair a complexidade do hardware físico através de representações virtuais \cite{mafamane_study_2021}, e (3) garantir a gestão unificada do \textit{streaming} de vídeo \cite{schwarzer_ial_2021}. A adoção de uma estrutura modular, em detrimento de uma abordagem monolítica rígida, assegura que o sistema possa acomodar novos protocolos e dispositivos futuramente \cite{dobaj_microservice_2018}, garantindo que a aplicação central permaneça desacoplada da volatilidade do hardware e assegurando a sua escalabilidade a longo prazo.


\section{LRRQ2 - Protocolos de Comunicação Segura e VPN}
\label{sec:lrrq2_vpn}

Nesta secção, a literatura é revista de forma a responder à questão de investigação ''Qual o estado da arte em protocolos de comunicação segura e \glspl{VPN} para acesso remoto e \textit{streaming}''.

\subsection{Processo de Pesquisa}

A questão de investigação foi estruturada de acordo com o modelo PICOCS apresentado na Tabela \ref{tab:lrrq2_picocs}.

\begin{table}[H]
	\centering
	\caption{Modelo PICOCS para a LRRQ2}
	\label{tab:lrrq2_picocs}
	\begin{tabular}{p{0.15\linewidth}p{0.35\linewidth}p{0.2\linewidth}p{0.2\linewidth}}
		\toprule
		\textbf{PICOCS} & \textbf{Parte da RQ} & \textbf{I/E} & \textbf{Sub string} \\
		\midrule
		\textbf{P}  & Estudos envolvendo acesso remoto a dispositivos em redes não controladas & I - Estudos que considerem redes não controladas \newline E - Anteriores a 2018 & >=2018; Remote access; Uncontrolled networks; Hostile networks \\
		\textbf{I}  & Protocolos de comunicação segura, \glspl{VPN} e mecanismos de túnel & I - Estudos sobre protocolos de segurança e túneis \newline E - Soluções proprietárias fechadas & Secure protocols; VPN; Tunneling; Encryption; Secure communication \\
		\textbf{C}  & --- & --- & --- \\
		\textbf{O} & Estado da arte em protocolos e desempenho de \glspl{VPN} & I - Estudos que analisem ou comparem protocolos \newline I - Análises de desempenho & Protocol comparison; VPN performance; Security analysis; Streaming protocols \\
		\textbf{C} & Comunicações de rede e segurança de sistemas distribuídos & I - Académico \newline I - Indústria & Network security; Distributed systems; Secure streaming \\
		\textbf{S}  & Estudos de caso, análises comparativas e propostas de protocolos & I - Case Studies \newline I - Comparative analysis \newline E - Opiniões sem validação & Case study; Comparative study; Protocol proposal; Performance evaluation \\
		\bottomrule
	\end{tabular}
\end{table}

A Tabela \ref{tab:lrrq2_picocs} inclui os critérios de inclusão e exclusão (I/E) relacionados com cada parte da questão de investigação, bem como as possíveis \textit{sub-strings} utilizadas para conduzir a pesquisa nas bases de dados científicas. Apenas estudos de 2018 em diante que contemplem protocolos de comunicação segura, \glspl{VPN} ou mecanismos de túnel para acesso remoto e \textit{streaming} foram considerados.


A pesquisa foi conduzida em três bases de dados principais: IEEE Xplore (62 registos), ACM Digital Library (15 registos) e ScienceDirect (2 registos), totalizando 79 registos identificados através de pesquisa sistemática. Adicionalmente, foi identificado 1 registo através de outras fontes, o mesmo identificado para a outra questão \cite{resende_sistema_2025}, resultando num total de 80 registos na fase de identificação.

Após a verificação de duplicados, não foram encontrados registos duplicados, mantendo-se os 80 registos para análise na fase de \textit{screening} através da leitura de títulos e resumos. Desta análise, 52 registos foram excluídos por não cumprirem os critérios de inclusão, nomeadamente por serem irrelevantes para o tema de \glspl{VPN}, protocolos seguros ou por falta de foco em acesso remoto. Os 28 registos restantes foram avaliados em texto completo, dos quais 9 foram excluídos por tecnologia obsoleta, falta de detalhes de implementação ou por estarem fora do âmbito do estudo. O processo resultou na inclusão final de 19 estudos na revisão qualitativa, conforme ilustrado na Figura \ref{fig:lrrq2_prisma}.

\begin{figure}[H]
\centering
\includegraphics[width=\textwidth]{ch2/assets/LRRQ2-prisma.png}
\caption{Fluxo PRISMA para a LRRQ2}
\label{fig:lrrq2_prisma}
\end{figure}

O rigor na filtragem aplicada, reduzindo de 80 para 19 estudos incluídos, garantiu que apenas trabalhos com protocolos de comunicação segura relevantes, detalhados e validadas fossem analisados. Comparativamente à LRRQ1, que identificou 435 registos resultando em 11 estudos incluídos (\percentage{2,5} de taxa de aproveitamento), a presente questão apresentou uma taxa de aproveitamento significativamente superior (\percentage{23,8}), indicando que a literatura sobre protocolos de comunicação segura e \glspl{VPN} apresenta uma maior concentração de artigos com conteúdo interessante e utilizável para fundamentar as decisões de segurança e comunicação do \gls{FUSE}.

\subsection{Discussão}

A análise dos 19 estudos incluídos permitiu traçar uma evolução clara nas tecnologias de acesso remoto seguro, partindo de soluções \textit{legacy} e centralizadas para arquiteturas modernas, descentralizadas e baseadas em túneis de alta performance. Esta subsecção sintetiza as descobertas literárias, organizando-as em quatro vetores fundamentais para o desenho da solução \gls{FUSE}.

\subsubsection{O Desafio da Conectividade em Redes Não Controladas}

Um consenso transversal na literatura é a impraticabilidade e insegurança da exposição direta de dispositivos \gls{IoT} à Internet. O estudo \cite{bugeja_investigation_2018} demonstra que o método tradicional de ''Port Forwarding'' expõe vulnerabilidades críticas de firmware, defendendo o uso de túneis como única defesa viável. Contudo, a criação destes túneis enfrenta barreiras estruturais nas redes modernas. A investigação \cite{boonprasert_low-cost_2024} valida o problema do \gls{CGNAT} em redes móveis (4G/5G), onde os \glspl{ISP} não alocam endereços \gls{IP} públicos aos dispositivos, tornando impossível o acesso direto. A solução validada em \cite{hritcan_exposing_2024} reside na inversão do modelo de conexão: é o dispositivo em Edge que deve iniciar o túnel para um ponto de encontro externo, ou utilizar redes \gls{P2P} (como o Tailscale) para garantir a travessia de \gls{NAT} sem configurações complexas de firewall.

\subsubsection{Limitações dos Protocolos Legacy e Comparação TCP vs. UDP}

A literatura estabelece protocolos como OpenVPN e IPSec como referências sólidas para a transmissão de dados de sensores  \cite{fan_design_2019, guarino_data_2025}. No entanto, a sua aplicação a fluxos de vídeo em tempo real apresenta limitações severas. 
O artigo \cite{asim_sect_2025} destaca o problema crítico do "TCP-over-TCP meltdown'': o encapsulamento de tráfego de vídeo (que beneficia da natureza \textit{fire-and-forget} do \gls{UDP}) dentro de túneis baseados em \gls{TCP} (comuns em \glspl{VPN} baseadas em \gls{SSL}) provoca um ciclo vicioso de retransmissões redundantes, resultando em latência exponencial sob redes instáveis.
Para além da performance, o estudo \cite{xue_openvpn_2024} alerta para a vulnerabilidade de segurança: o tráfego OpenVPN possui padrões de dados identificáveis, como tamanho de pacotes e tempos de resposta, permitindo que \glspl{ISP} reconheçam e bloqueiem a conexão, mesmo quando encriptada.

\subsubsection{A Ascensão do WireGuard e Túneis de Alta Performance}

Como resposta às limitações supracitadas, o protocolo WireGuard emerge na literatura analisada como o novo estado da arte para comunicações seguras em dispositivos com recursos limitados. Os trabalhos \cite{hohmann_bridge_2021} e \cite{dang_ewdc_2025} fornecem evidência empírica de que o WireGuard introduz uma latência mínima (na ordem dos microsegundos) e um \textit{overhead} de processamento significativamente inferior aos antecessores. A sua eficiência em hardware \textit{low-cost} (como Raspberry Pi) é validada no estudo \cite{bezenk_remote_2024}, tornando-o ideal para a arquitetura Edge do \gls{FUSE}.
Mais relevante ainda, \cite{resende_sistema_2025} valida a performance do WireGuard em \gls{SBC} para streaming de vídeo, embora utilize uma topologia de rede dependente de encaminhamento de portas (\textit{Port Forwarding}) que o \gls{FUSE} pretende superar. O artigo \cite{tong_daiottalk_2025} reforça a abordagem de evolução para arquiteturas descentralizadas, onde a comunicação ocorre diretamente entre o produtor e o consumidor de dados (P2P), demonstrando ser significativamente mais eficiente do que modelos que dependem de um servidor central, apresentando ganhos de desempenho na ordem de 3x a 5x.

\subsubsection{Análise Crítica}

A síntese dos estudos permite validar as decisões arquiteturais do \gls{FUSE} por oposição às abordagens tradicionais. A literatura suporta a implementação de um ''Secure Gateway'' em Edge \cite{garcia_containerized_2023, k_efficient_2024} como ponto central de segurança. Embora estes estudos demonstrem o potencial desta arquitetura para a implementação de pipelines de \gls{CI/CD} e atualizações de firmware (\gls{FOTA}), a presente dissertação foca-se exclusivamente na transmissão segura de vídeo, considerando a gestão complexa de frota um tema adjacente que, pela sua complexidade, supera o âmbito deste trabalho, apesar de tecnicamente viável na arquitetura proposta. Contrariando a intuição de que a encriptação degrada a performance, o estudo \cite{avanzato_enhancing_2023} sugere que o encapsulamento em túneis modernos pode, de facto, melhorar a qualidade de experiência em redes instáveis.

Relativamente ao risco de deteção por \gls{DPI} levantado em \cite{xue_openvpn_2024}, o \gls{FUSE} assume uma posição pragmática: embora a ofuscação total seja desejável, esta impõe frequentemente penalizações severas ao débito de dados. Assim, o projeto opta pelo WireGuard para priorizar a performance e a fluidez do vídeo em detrimento de técnicas de ofuscação pesada, assumindo um compromisso calculado entre latência mínima e furtividade absoluta.

Importa referir que quatro dos estudos selecionados na filtragem \gls{PRISMA} não foram aprofundados na discussão principal, servindo antes como ''Baseline da Indústria'' que contextualiza a inovação do FUSE. Enquanto a literatura tradicional valida abordagens baseadas em \textit{fog computing} com \textit{blockchain} \cite{trabelsi_fog_2024} ou arquiteturas centralizadas clássicas (IPSec/DMVPN) \cite{qaraqe_publicvision_2024}, o estado da arte aponta para soluções mais leves e descentralizadas. Similarmente, os estudos \cite{maceda_scada_2022} e \cite{hussain_securing_2019} documentam o uso de OpenVPN para sistemas \textit{legacy} (SCADA) e a sua otimização, reforçando a prevalência histórica destas soluções que o presente projeto opta por superar através da adoção integral do WireGuard.

Concretamente, a arquitetura do \gls{FUSE} materializa-se na disponibilização de cada câmara (ou conjunto de câmaras) em conjunto com um \textbf{\gls{SBC}}, especificamente um Raspberry Pi. Este dispositivo atuará como um gateway de rede, controlando o tráfego e estabelecendo o encaminhamento através de uma \gls{VPN} gerida pelo \textbf{Tailscale}. Embora o Tailscale utilize um servidor de coordenação centralizado (\gls{SaaS}) como prova de conceito, a arquitetura modular permite a transição futura para implementações \textit{self-hosted} (como o Headscale) para garantir a total posse e confidencialidade dos dados. Desta forma, o servidor central consegue agregar e controlar o \textit{streaming} de vídeo de todas as câmaras dispersas, garantindo a segurança sem expor os dispositivos diretamente à rede pública.


\section{Estado da Arte em Visão Computacional}
\label{sec:estado_arte_visao}

Relativamente à terceira sub-questão, a RQ1.3, referente à automatização da análise de vídeo, optou-se por uma abordagem de Revisão Narrativa e Exploratória, como descrita por Maria J. Grant e Andrew Booth em \cite{Grant2009} como ''State-of-the-art review''. Esta modalidade metodológica caracteriza-se pela sua flexibilidade na análise crítica da literatura atual, permitindo identificar conceitos-chave, padrões arquiteturais e soluções técnicas emergentes sem a rigidez protocolar de uma revisão sistemática e formal.

Esta opção justifica-se pela vertiginosa evolução dos modelos de \gls{AI} e pela necessidade de analisar não apenas literatura académica clássica, mas também documentação técnica de modelos recentes e \textit{benchmarks} da indústria. A análise encontra-se segmentada em três domínios fundamentais que correspondem às fases da pipeline de processamento proposta para o \gls{FUSE}:

\begin{enumerate}
	\item Deteção de Movimento (Fase 1)
	\item Deteção de Objetos (Fase 2)
	\item Visual-Language Models - \glspl{VLM} (Fase 3)
\end{enumerate}

Apesar da natureza exploratória, esta pesquisa manterá o rigor científico na seleção de fontes, priorizando publicações de menos de 1 ano para os domínios de evolução rápida (Fase 2 e Fase 3), dado o exponencial e recente crescimento tecnológico da área, e repositórios open-source com forte validação comunitária. A pesquisa foi orientada pelas seguintes \textit{keywords}, agrupadas por domínio:

\begin{itemize}
	\item \textbf{Fase 1 (Pré-processamento):} Background Subtraction, Motion Detection Algorithms, Frame Differencing efficiency, Video Activity Detection.
	\item \textbf{Fase 2 (Classificação):} Real-time Object Detection, \gls{YOLO} architecture, One-stage detectors, \gls{CNN} inference optimization, Edge \gls{AI}.
	\item \textbf{Fase 3 (Extração de Atributos):} Vision-Language Models (\glspl{VLM}), Multimodal \gls{AI}, Zero-Shot Learning, Open-vocabulary detection, Visual Question Answering.
\end{itemize}

\subsection{Processamento em Edge vs Centralizado}
\label{subsec:limitacoes_edge}

A arquitetura de sistemas de videovigilância inteligentes tem oscilado historicamente entre o processamento em Edge (\textit{Edge Computing}) e na nuvem/servidor central (\textit{Cloud Computing}). Embora a tendência recente favoreça o \textit{Edge AI} para reduzir a latência, a literatura mais atual revela também barreiras intransponíveis para a aplicação de modelos de IA avançados em hardware \textit{legacy} ou de baixo custo.

O estudo \cite{bai_decade_2025} demonstra extensivamente que a execução de modelos de visão computacional em Edge exige recursos de memória consideráveis. Mesmo modelos leves como o MobileNet requerem mais de 1GB de memória para processar imagens de resolução média (512x512) a 15fps, o que inviabiliza a sua utilização em câmaras \gls{IP} comuns ou dispositivos IoT sem hardware dedicado. Complementarmente, o trabalho \cite{pecolt_personal_2025} validou experimentalmente que dispositivos Edge populares (como Raspberry Pi), mesmo utilizando algoritmos leves, introduzem latências de reconhecimento na ordem dos 10.5 segundos, um valor inaceitável para aplicações de segurança em tempo real.

Em contrapartida, o estudo \cite{lu_real-time_2025} propõe e valida uma arquitetura centralizada baseada em \textit{Cloud/Server}, demonstrando que esta abordagem não só permite a execução de modelos mais complexos, como resulta numa redução de \percentage{38} no custo total de propriedade (\gls{TCO}) e num aumento de 2.5x na capacidade de processamento face a sistemas tradicionais descentralizados. Estes dados fundamentam a decisão arquitetural do \gls{FUSE} de rejeitar o processamento pesado em Edge, optando por uma centralização inteligente.

\subsection{Deteção de Movimento e Filtragem Temporal (Fase 1)}
\label{subsec:deteção_movimento}

Dada a inviabilidade de processar continuamente streams de vídeo de alta resolução com modelos de Deep Learning pesados, a primeira fase da pipeline impõe-se como um mecanismo de filtragem eficiente. A literatura valida a estratégia de ''Event-Driven Surveillance'', onde algoritmos leves de deteção de movimento atuam como "portão" de entrada para o sistema. Por outras palavras, apenas ser processado determinado vídeo correspondente a uma porção de uma \textit{stream} que resultou da deteção de movimento na mesma.

O estudo \cite{sultana_iot-guard_2019}, embora de 2019, permanece relevante dada a maturidade das técnicas de filtragem de movimento, demonstrando que uma arquitetura orientada a eventos, utilizando algoritmos simples de subtração de fundo ou diferenciação de frames em Edge, permite reduzir o consumo de \gls{CPU} em cerca de \percentage{60} e poupar mais de \percentage{99} em largura de banda e armazenamento em comparação com a transmissão contínua. Esta abordagem, chamada de ''Temporal Pruning'', também defendida pelo estudo mais recente \cite{bai_decade_2025} como estado da arte em eficiência, valida a Fase 1 do \gls{FUSE}: o uso de métodos clássicos e computacionalmente baratos para eliminar frames estáticos, garantindo que os recursos do servidor, nomeadamente a(s) \gls{GPU}, são alocados exclusivamente a segmentos de vídeo com atividade relevante.

\subsection{Deteção e Classificação de Objetos (Fase 2)}
\label{subsec:deteção_objetos}

Para a análise dos segmentos de vídeo filtrados, a necessidade de processamento em tempo real exige modelos de deteção de objetos que maximizem o compromisso entre velocidade de inferência e precisão. A família de modelos \gls{YOLO} mantém-se como a referência indisputada neste domínio.

Estudos comparativos recentes, como o estudo \cite{khanam_comparative_2025}, destacam o \textbf{YOLOv11}, lançado no final de 2024, como a evolução ideal para ambientes de servidor de alto desempenho. Na sua avaliação, o YOLOv11 superou significativamente as versões anteriores (v8 e v9), atingindo tempos de inferência de apenas 7.7ms por imagem (quase o dobro da rapidez do YOLOv8, que registou 15.9ms) mantendo a maior precisão média (mAP@0.5 de \percentage{93,4}). Estes resultados são suportados pelos relatórios técnicos oficiais da Ultralytics \cite{jocher_ultralytics_2024}, que indicam que o YOLOv11m atinge um novo estado da arte ao oferecer maior precisão com \percentage{22} menos parâmetros que o seu predecessor. Esta redução de complexidade computacional é crítica para a escalabilidade do \gls{FUSE}, permitindo processar múltiplos streams simultâneos num único servidor.

\subsection{Modelos de Visão-Linguagem (\glspl{VLM}) e Identificação Alvo (Fase 3)}
\label{subsec:vlm}

A deteção de objetos clássica (Fase 2) identifica a presença de entidades genéricas (e.g., ''carro vermelho''), mas é insuficiente para confirmar atributos específicos definidos pelo utilizador (e.g., ''matrícula 43-RD-45''). A fronteira da investigação situa-se na integração de \textit{Vision-Language Models} (\glspl{VLM}) como uma camada de verificação semântica em cascata.

O estudo \cite{xia_enhancing_2025} argumenta que a simples deteção falha em distinguir instâncias específicas devido a oclusões ou semelhanças visuais. Para resolver esta limitação, o \gls{FUSE} adota uma arquitetura onde a Fase 3 atua sobre os recortes da Fase 2 para validar atributos textuais ou visuais complexos. O trabalho \cite{bang_reasoning-augmented_2025} valida a viabilidade desta abordagem ao demonstrar que modelos da família \textbf{Qwen2.5-VL}, equipados com mecanismos de raciocínio, conseguem realizar \gls{OCR} e compreensão de contexto com precisão superior a métodos tradicionais, permitindo, por exemplo, ler uma matrícula ou identificar um logótipo específico.

Complementarmente, o estudo \cite{chen_florence-vl_2025} reforça a necessidade de modelos generativos robustos (como o Florence-2) para tarefas de \textit{Grounding}, ou seja, localizar com precisão o atributo alvo dentro da imagem. Esta capacidade de ''raciocinar'' sobre o conteúdo visual valida a necessidade de hardware de servidor robusto, uma vez que a execução destes modelos exige memória e capacidade de computação muito superiores às de um detetor convencional, sendo inviável em dispositivos Edge.

\subsection{Armazenamento e Pesquisa Forense}
\label{subsec:storage_future}

Embora o âmbito do \gls{MVP} do \gls{FUSE} privilegie a deteção e notificação de eventos, a revisão da literatura permitiu identificar estratégias emergentes para o armazenamento de dados que potenciam a utilidade forense a longo prazo. A simples acumulação de metadados, sem uma estratégia de indexação semântica, limita a capacidade futura de realizar pesquisas complexas sobre o histórico de eventos.

Neste contexto, a investigação em \cite{pham_deep_2025} e \cite{j_advanced_2025} aponta para a insuficiência das bases de dados relacionais clássicas face à complexidade dos dados de vídeo modernos. Estes trabalhos validam, respetivamente, o uso de Grafos e de arquiteturas \gls{RAG} com \textit{Embeddings} Vetoriais como o estado da arte para transformar outputs do modelo da Fase 3 da pipeline de análise automática em registos pesquisáveis.

A identificação destas tecnologias durante o estudo do estado da arte valida a pertinência da extração de atributos proposta, garantindo que os dados gerados pelo \gls{FUSE} possuem a riqueza semântica necessária. Tal permite que, numa fase de desenvolvimento posterior ao \gls{MVP}, o sistema possa evoluir de uma ferramenta reativa de alertas para uma plataforma de investigação forense capaz de suportar pesquisas em linguagem natural.

\section{Conclusão do Estado da Arte}
\label{sec:conclusao_estado_arte}

A revisão sistemática e exploratória da literatura permitiu identificar lacunas críticas e validar a arquitetura proposta para o \gls{FUSE} em torno de três pilares fundamentais: interoperabilidade, conetividade segura e inteligência artificial distribuída.

Primeiro, relativamente à heterogeneidade de dispositivos (LRRQ1), a literatura \cite{barros_iot_2019,mesmoudi_design_2018} demonstra inequivocamente que a integração direta de câmaras \gls{IoT} é insustentável. A solução validada reside na implementação de uma \textbf{Camada de Abstração Modular} baseada em microserviços e no padrão ''Digital Twin'' \cite{mafamane_study_2021}, que isola a aplicação central da complexidade dos drivers proprietários e normaliza o \textit{streaming} de vídeo para formatos web standard \cite{schwarzer_ial_2021}.

Segundo, no domínio da segurança em redes não controladas (LRRQ2), o estado da arte rejeita as abordagens centralizadas e baseadas em \gls{TCP} (como OpenVPN) devido à latência excessiva e problemas de ''TCP meltdown'' \cite{asim_sect_2025}. A arquitetura do \gls{FUSE} alinha-se com a tendência de descentralização, adotando túneis \textbf{WireGuard} iniciados na Edge \cite{hohmann_bridge_2021,bezenk_remote_2024} e topologias \gls{P2P} (como demonstrado por \cite{tong_daiottalk_2025}) para garantir a soberania dos dados e contornar barreiras de \gls{NAT} sem depender de servidores de retransmissão lentos.

Por fim, a análise das tecnologias de visão computacional confirma a inviabilidade de executar modelos de \gls{AI} modernos em hardware \textit{legacy} em Edge \cite{bai_decade_2025,pecolt_personal_2025}. Esta limitação valida a decisão arquitetural de centralizar a inteligência: embora a Edge possa atuar como um filtro preliminar simples (como a deteção de movimento \cite{sultana_iot-guard_2019}), a inferência semântica é delegada num servidor central. Neste servidor, a combinação do estado da arte em deteção de objetos (\textbf{YOLOv11}) com a nova fronteira de \textbf{Vision-Language Models} (Qwen2.5-VL) permite não só detetar, mas ''raciocinar'' sobre o vídeo para extrair atributos específicos (e.g., matrículas), viabilizando uma plataforma de segurança verdadeiramente inteligente e pesquisável. Importa notar que, embora a literatura valide o Qwen2.5-VL, o projeto antecipa a adoção do recém-anunciado \textbf{Qwen3-VL}, assumindo, até prova em contrário, que a evolução natural da família trará ganhos de desempenho e raciocínio fundamentais para o \gls{FUSE}.
