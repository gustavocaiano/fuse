% Chapter 2

\chapter{Estudo do Estado da Arte}
\label{chap:estado_da_arte}

Este capítulo apresenta a revisão de literatura fundamentada no âmbito do projeto, analisando o estado atual das tecnologias críticas para o desenvolvimento do \gls{FUSE}. O objetivo principal desta revisão é responder às sub-questões operacionais previamente identificadas, nomeadamente a RQ1.1 e a RQ1.2, estabelecendo uma base teórica sólida para as decisões de arquitetura e segurança.

Adicionalmente, e dada a natureza aplicada da RQ1.3, é conduzido um estudo técnico aprofundado sobre os paradigmas atuais de visão computacional. Este estudo visa não apenas identificar o estado da arte, mas também comparar padrões, pipelines de processamento e modelos de \gls{AI} passíveis de integração na plataforma.

\section{Processo de Investigação}
\label{sec:processo_investigacao}

O Processo de investigação foi guiado pelo mapeamento das duas primeiras sub-questões da presente dissertação em questões de pesquisa com foco na revisão literária, conforme descrito na Tabela \ref{tab:questoes_pesquisa}.

\begin{table}[H]
\centering
\caption{Questões de Pesquisa Orientadas à Revisão Literária}
\label{tab:questoes_pesquisa}
\begin{tabular}{llp{5cm}p{5cm}}
\toprule
\textbf{RQ ID} & \textbf{LRRQ ID} & \textbf{Description} & \textbf{Tópicos e Keywords de pesquisa} \\
\midrule
RQ1.1 & LRRQ1 & Quais são as arquiteturas de referência e padrões de design de software descritos na literatura para a abstração e interoperabilidade de dispositivos \gls{IoT} heterogéneos? & \gls{IoT} Interoperability, Hardware Abstraction Layer, Middleware patterns, \gls{ONVIF} standardization, Heterogeneous device integration. \\
RQ1.2 & LRRQ2 & Qual o estado da arte em protocolos de comunicação segura e \glspl{VPN} para acesso remoto e \textit{streaming}? & Secure Tunneling Protocols, \gls{VPN} performance analysis, Streaming encryption, \gls{NAT} Traversal, \gls{ZTNA}. \\
\bottomrule
\end{tabular}
\end{table}

Cada questão de pesquisa será enquadrada de acordo com o modelo \gls{PICOCS} \cite{Alberto2015}, garantindo o rigor na seleção das fontes utilizadas. A informação será depois selecionada seguindo o fluxo \gls{PRISMA} \cite{Moher2010}. O processo inclui a definição de keywords de pesquisa, a aplicação estrita de critérios de inclusão e exclusão, seguida de uma filtragem em etapas e, finalmente, uma discussão crítica sobre a aplicabilidade dos estudos selecionados para a arquitetura do \gls{FUSE}.

Como fator inclusivo de seleção, selecionou-se, por exemplo, a data de publicação posterior a 2018. A escolha deste intervalo visa garantir que as arquiteturas, frameworks e algoritmos analisados representam o atual estado da arte, evitando a adoção de paradigmas que, embora válidos no passado, não refletem as necessidades de desempenho e escalabilidade dos sistemas modernos. Foram privilegiados artigos revistos por pares, normas técnicas e literatura técnica amplamente reconhecida. Excluíram-se estudos puramente teóricos sem aplicação prática ao domínio da videovigilância ou da integração de dispositivos, tal como trabalhos relativos a soluções proprietárias fechadas sem documentação técnica pública suficiente.


\section{LRRQ1 - Arquiteturas de Abstração e Interoperabilidade IoT}
\label{sec:lrrq1_iot}

Nesta secção, a literatura é revista de forma a responder à questão de investigação "Quais são as arquiteturas de referência e padrões de design de software descritos na literatura para a abstração e interoperabilidade de dispositivos IoT heterogéneos?".

\subsection{Processo de Pesquisa}

A questão de investigação foi estruturada de acordo com o modelo PICOCS apresentado na Tabela \ref{tab:lrrq1_picocs}.

\begin{table}[H]
	\centering
	\caption{Modelo PICOCS para a LRRQ1}
	\label{tab:lrrq1_picocs}
	\begin{tabular}{p{0.15\linewidth}p{0.35\linewidth}p{0.2\linewidth}p{0.2\linewidth}}
		\toprule
		\textbf{PICOCS} & \textbf{Parte da RQ} & \textbf{I/E} & \textbf{Sub string} \\
		\midrule
		\textbf{P}  & Estudos envolvendo dispositivos IoT heterogéneos e CCTV & I - Estudos que considerem heterogeneidade \newline E - Anteriores a 2018 & >=2018; IoT; Heterogeneous; CCTV \\
		\textbf{I}  & Arquiteturas de software, Middleware e Camadas de Abstração & I - Estudos sobre padrões de design e abstração \newline E - Soluções proprietárias fechadas & Software Architecture; Middleware; Abstraction Layer; Design Patterns \\
		\textbf{C}  & --- & --- & --- \\
		\textbf{O} & Arquiteturas de referência e interoperabilidade & I - Estudos que propõem ou validam arquiteturas & Architecture; Interoperability; Framework \\
		\textbf{C} & Engenharia de Software e Sistemas Distribuídos & I - Académico \newline I - Indústria & Software Engineering; Distributed Systems \\
		\textbf{S}  & Estudos de caso e Propostas de Arquitetura & I - Case Studies \newline I - System Proposals \newline E - Opiniões sem validação & Case study; Proposal; Prototype \\
		\bottomrule
	\end{tabular}
\end{table}

A tabela inclui os critérios de inclusão e exclusão (I/E) relacionados com cada parte da questão de investigação, bem como as possíveis \textit{sub-strings} utilizadas para conduzir a pesquisa nas bases de dados científicas. Apenas estudos de 2018 em diante que contemplem arquiteturas de software ou middleware para integração de dispositivos heterogéneos foram considerados.

A pesquisa foi conduzida em três bases de dados principais: IEEE Xplore (282 registos), ACM Digital Library (94 registos) e ScienceDirect (59 registos), totalizando 435 registos identificados através de pesquisa sistemática. Adicionalmente, foi identificado 1 registo \cite{resende_sistema_2025} através de outras fontes, resultando num total de 436 registos na fase de identificação. 

Após a remoção de duplicados (1 registo), foram analisados 435 registos na fase de screening através da leitura de títulos e resumos. Desta análise, 412 registos foram excluídos por não cumprirem os critérios de inclusão, nomeadamente por focarem-se puramente em algoritmos de visão computacional sem arquitetura de sistema, abordarem domínios como agricultura ou saúde sem componente de vídeo, ou simplemsmente não se adequar ao tema de integração de dispositivos \gls{IoT} ou de camadas de abstração em design de software. Os 23 registos restantes foram avaliados em texto completo, dos quais 12 foram excluídos por falta de detalhes de implementação, ausência de diagrama de arquitetura claro, ou por serem propostas teóricas sem validação prática. O processo resultou na inclusão final de 11 estudos na revisão qualitativa, conforme ilustrado na Figura \ref{fig:lrrq1_prisma}.

\begin{figure}[H]
\centering
\includegraphics[width=\textwidth]{ch2/assets/LRRQ1-prisma.png}
\caption{Fluxo PRISMA para a LRRQ1}
\label{fig:lrrq1_prisma}
\end{figure}

O rigor na filtragem aplicada, reduzindo de 435 para 11 estudos incluídos, garantiu que apenas trabalhos com arquiteturas de sistema relevantes, detalhadas e validadas fossem analisados, assegurando a qualidade e relevância da revisão de literatura para fundamentar as decisões arquiteturais do \gls{FUSE}.

\subsection{Discussão}

A análise dos 11 estudos incluídos revela uma convergência clara na literatura: a integração direta de dispositivos \gls{IoT} heterogéneos sem uma camada intermediária resulta em sistemas monolíticos e de difícil manutenção.A diversidade de protocolos (\gls{RTSP}, \gls{HTTP}, \gls{MQTT}), algoritmos de codificação de vídeo (H.264, H.265, MJPEG) e especificidades de fabricantes exige, invariavelmente a implementação de uma camada de abstração que atue como tradutor universal. Como observa \cite{mesmoudi_design_2018}, a ausência desta camada força o sistema central a lidar com a complexidade de hardware, violando princípios de desacoplamento. A literatura valida que a solução reside numa arquitetura que isole a lógica de negócio das particularidades dos dispositivos.

\subsubsection{Padrões Arquiteturais: De Gateways a Microserviços Agnósticos}

A evolução dos padrões arquiteturais na literatura aponta para um distanciamento de \textit{gateways }simples em direção a microserviços inteligentes. Inicialmente, soluções como o IoTM2B \cite{barros_iot_2019} e Smart Gateway \cite{mesmoudi_design_2018} validaram o uso de um ponto central para tradução de protocolos (ex: converter \gls{MQTT}/\gls{CoAP} para \gls{HTTP}). No entanto, \cite{barros_iot_2019} identifica uma limitação crítica: a normalização para \gls{HTTP} POST, embora eficaz para telemetria escalar (temperatura), introduz um gargalo de desempenho inaceitável para \textit{streaming} de vídeo contínuo.

A resposta a esta limitação encontra-se na adoção de microserviços com o padrão \textit{Mediator}, como proposto por \cite{dobaj_microservice_2018}. Esta abordagem introduz uma Camada de Abstração de Dados dedicada que desacopla a lógica de serviço da comunicação com o dispositivo. Mais relevante para o conceito do \gls{FUSE} é a introdução do padrão ``Access Mapper'' ou ``Digital Twin'' por \cite{mafamane_study_2021}. Neste modelo, a aplicação central interage exclusivamente com um ``Dispositivo Virtual'' estandardizado, tornando-se completamente agnóstica ao hardware físico. Este é o fundamento teórico para um componente da solução a desenvolver, ou seja, uma camada que encapsula a complexidade do driver proprietário e expõe uma interface limpa e uniforme.

\subsubsection{Abstração de Media e Normalização de Streaming}

A abstração em sistemas de videovigilância exige também a normalização do próprio fluxo de vídeo. A conexão direta entre uma aplicação web e câmaras de vigilância apresenta desafios técnicos significativos devido à natureza dos fluxos de vídeo brutos e algoritmos de codificação de vídeo utilizados. A literatura, nomeadamente \cite{schwarzer_ial_2021}, suporta a implementação de uma camada intermédia responsável pela ingestão e gestão destes fluxos, expondo-os posteriormente num formato standardizado e otimizado para consumo web (como \gls{HLS} ou \gls{WebRTC}).

Esta abordagem desacopla a aplicação cliente da complexidade de conexão direta às câmaras: o cliente consome sempre um fluxo normalizado a partir da camada de abstração. Embora \cite{schwarzer_ial_2021} e \cite{barra_exploring_2024} discutam também o processamento semântico destes fluxos na Edge (análise de vídeo), este tópico específico de automação e visão computacional será aprofundado na Secção \ref{sec:estado_arte_visao}. Para a LRRQ1, o contributo essencial é a arquitetura que garante a estabilidade e abstração do \textit{streaming}.

\subsubsection{Escalabilidade e Adaptabilidade à Rede}

A viabilidade futura do sistema depende da sua capacidade de escalar para novos dispositivos e adaptar-se a redes instáveis. \cite{mesmoudi_design_2018} demonstra a importância do \textit{ThingDiscovery} automatizado para detetar novas câmaras sem configuração manual, um requisito essencial para a usabilidade. No que toca à rede, \cite{george_distributed_2019} introduz o conceito vital de ``Latency-Accuracy Trade-off'', onde o middleware ajusta dinamicamente parâmetros (como resolução) face à congestão da rede. A validação empírica desta topologia distribuída é fornecida por \cite{gomez_strategies_2023}, que comprova uma redução de latência de até 98\% em arquiteturas Edge Computing comparativamente à Cloud pura para serviços críticos. Adicionalmente, \cite{immich_multi-tier_2019} demonstra as vantagens de uma arquitetura baseada em cadeias de módulos independentes. Esta modularidade permite que componentes distintos (como a ingestão de vídeo, a transcodificação e a análise) operem de forma independente e encadeada. Para o \gls{FUSE}, isto significa que é possível atualizar ou substituir módulos individuais sem comprometer a estabilidade do sistema base.

\subsubsection{Análise Crítica}

Apesar da diversidade de soluções, a revisão identifica lacunas relevantes para o contexto do \gls{FUSE}. Primeiro, a maioria das arquiteturas foca-se em telemetria (sensores), negligenciando os requisitos específicos de largura de banda do vídeo \cite{barros_iot_2019}. Segundo, embora existam padrões sintáticos, \cite{muniz_pragmatic_2019} alerta para a falta de ``Interoperabilidade Pragmática'', ou seja, a capacidade dos dispositivos comunicarem a intenção do dado (ex: urgência de um alerta) e não apenas o conteúdo.

O trabalho de \cite{resende_sistema_2025} valida a viabilidade de usar plataformas open-source como o ZoneMinder para orquestrar estes fluxos em hardware modesto, atuando como um middleware de abstração eficaz. No entanto, o estudo alerta para os limites de performance quando se tenta realizar processamento analítico pesado no mesmo \textit{node}, reforçando a necessidade de uma arquitetura eficiente e distribuída.

\subsubsection{Conclusão}

A revisão da literatura fundamenta, de forma inequívoca, que a arquitetura do \gls{FUSE} deve assentar numa \textbf{Camada de Abstração Modular localizada}. Esta camada tem como responsabilidades críticas: (1) normalizar a comunicação com dispositivos heterogéneos para uma interface agnóstica \cite{barros_iot_2019,mesmoudi_design_2018}, (2) abstrair a complexidade do hardware físico através de representações virtuais \cite{mafamane_study_2021}, e (3) garantir a gestão unificada do \textit{streaming} de vídeo \cite{schwarzer_ial_2021}. A adoção de uma estrutura modular, em detrimento de uma abordagem monolítica rígida, assegura que o sistema possa acomodar novos protocolos e dispositivos futuramente \cite{dobaj_microservice_2018}, garantindo que a aplicação central permaneça desacoplada da volatilidade do hardware e assegurando a sua escalabilidade a longo prazo.


\section{LRRQ2 - Protocolos de Comunicação Segura e VPN}
\label{sec:lrrq2_vpn}

Nesta secção, a literatura é revista de forma a responder à questão de investigação ''Qual o estado da arte em protocolos de comunicação segura e \glspl{VPN} para acesso remoto e \textit{streaming}''.

\subsection{Processo de Pesquisa}

A questão de investigação foi estruturada de acordo com o modelo PICOCS apresentado na Tabela \ref{tab:lrrq2_picocs}.

\begin{table}[H]
	\centering
	\caption{Modelo PICOCS para a LRRQ2}
	\label{tab:lrrq2_picocs}
	\begin{tabular}{p{0.15\linewidth}p{0.35\linewidth}p{0.2\linewidth}p{0.2\linewidth}}
		\toprule
		\textbf{PICOCS} & \textbf{Parte da RQ} & \textbf{I/E} & \textbf{Sub string} \\
		\midrule
		\textbf{P}  & Estudos envolvendo acesso remoto a dispositivos em redes não controladas & I - Estudos que considerem redes não controladas \newline E - Anteriores a 2018 & >=2018; Remote access; Uncontrolled networks; Hostile networks \\
		\textbf{I}  & Protocolos de comunicação segura, \glspl{VPN} e mecanismos de túnel & I - Estudos sobre protocolos de segurança e túneis \newline E - Soluções proprietárias fechadas & Secure protocols; VPN; Tunneling; Encryption; Secure communication \\
		\textbf{C}  & --- & --- & --- \\
		\textbf{O} & Estado da arte em protocolos e desempenho de \glspl{VPN} & I - Estudos que analisem ou comparem protocolos \newline I - Análises de desempenho & Protocol comparison; VPN performance; Security analysis; Streaming protocols \\
		\textbf{C} & Comunicações de rede e segurança de sistemas distribuídos & I - Académico \newline I - Indústria & Network security; Distributed systems; Secure streaming \\
		\textbf{S}  & Estudos de caso, análises comparativas e propostas de protocolos & I - Case Studies \newline I - Comparative analysis \newline E - Opiniões sem validação & Case study; Comparative study; Protocol proposal; Performance evaluation \\
		\bottomrule
	\end{tabular}
\end{table}

A tabela inclui os critérios de inclusão e exclusão (I/E) relacionados com cada parte da questão de investigação, bem como as possíveis \textit{sub-strings} utilizadas para conduzir a pesquisa nas bases de dados científicas. Apenas estudos de 2018 em diante que contemplem protocolos de comunicação segura, \glspl{VPN} ou mecanismos de túnel para acesso remoto e \textit{streaming} foram considerados.


A pesquisa foi conduzida em três bases de dados principais: IEEE Xplore (62 registos), ACM Digital Library (15 registos) e ScienceDirect (2 registos), totalizando 79 registos identificados através de pesquisa sistemática. Adicionalmente, foi identificado 1 registo através de outras fontes, o mesmo identificado para a outra questão \cite{resende_sistema_2025}, resultando num total de 80 registos na fase de identificação.

Após a verificação de duplicados, não foram encontrados registos duplicados, mantendo-se os 80 registos para análise na fase de screening através da leitura de títulos e resumos. Desta análise, 52 registos foram excluídos por não cumprirem os critérios de inclusão, nomeadamente por serem irrelevantes para o tema de \glspl{VPN}, protocolos seguros ou por falta de foco em acesso remoto. Os 28 registos restantes foram avaliados em texto completo, dos quais 9 foram excluídos por tecnologia obsoleta, falta de detalhes de implementação ou por estarem fora do âmbito do estudo. O processo resultou na inclusão final de 19 estudos na revisão qualitativa, conforme ilustrado na Figura \ref{fig:lrrq2_prisma}.

\begin{figure}[H]
\centering
\includegraphics[width=\textwidth]{ch2/assets/LRRQ2-prisma.png}
\caption{Fluxo PRISMA para a LRRQ2}
\label{fig:lrrq2_prisma}
\end{figure}

O rigor na filtragem aplicada, reduzindo de 80 para 19 estudos incluídos, garantiu que apenas trabalhos com protocolos de comunicação segura relevantes, detalhados e validadas fossem analisados. Comparativamente à LRRQ1, que identificou 435 registos resultando em 11 estudos incluídos (2,5\% de taxa de aproveitamento), a presente questão apresentou uma taxa de aproveitamento significativamente superior (23,8\%), indicando que a literatura sobre protocolos de comunicação segura e \glspl{VPN} apresenta uma maior concentração de artigos com conteúdo interessante e utilizável para fundamentar as decisões de segurança e comunicação do \gls{FUSE}.

\subsection{Discussão}

A análise dos 19 estudos incluídos permitiu traçar uma evolução clara nas tecnologias de acesso remoto seguro, partindo de soluções \textit{legacy} e centralizadas para arquiteturas modernas, descentralizadas e baseadas em túneis de alta performance. Esta secção sintetiza as descobertas literárias, organizando-as em quatro vetores fundamentais para o desenho da solução \gls{FUSE}.

\subsubsection{O Desafio da Conectividade em Redes Não Controladas}

Um consenso transversal na literatura é a impraticabilidade e insegurança da exposição direta de dispositivos \gls{IoT} à Internet. \cite{bugeja_investigation_2018} demonstra que o método tradicional de ''Port Forwarding'' expõe vulnerabilidades críticas de firmware, defendendo o uso de túneis como única defesa viável. Contudo, a criação destes túneis enfrenta barreiras estruturais nas redes modernas. \cite{boonprasert_low-cost_2024} valida o problema do \gls{CGNAT} em redes móveis (4G/5G), onde os \glspl{ISP} não alocam endereços IP públicos aos dispositivos, tornando impossível o acesso direto (inbound). A solução validada por \cite{hritcan_exposing_2024} reside na inversão do modelo de conexão: é o dispositivo na Edge que deve iniciar o túnel para um ponto de encontro externo, ou utilizar redes \gls{P2P} (como o Tailscale) para garantir a travessia de \gls{NAT} sem configurações complexas de firewall.

\subsubsection{Limitações dos Protocolos \textit{Legacy} e Comparação TCP vs. UDP}

A literatura estabelece protocolos como OpenVPN e IPSec como referências sólidas para a transmissão de dados de sensores  \cite{fan_design_2019, guarino_data_2025}. No entanto, a sua aplicação a fluxos de vídeo em tempo real apresenta limitações severas. 
\cite{asim_sect_2025} destaca o problema crítico do "TCP-over-TCP meltdown'': o encapsulamento de tráfego de vídeo (que beneficia da natureza \textit{fire-and-forget} do \gls{UDP}) dentro de túneis baseados em \gls{TCP} (comuns em \glspl{VPN} baseadas em \gls{SSL}) provoca um ciclo vicioso de retransmissões redundantes, resultando em latência exponencial sob redes instáveis.
Para além da performance, \cite{xue_openvpn_2024} alerta para a vulnerabilidade de segurança: o tráfego OpenVPN possui padrões de dados identificáveis, como tamanho de pacotes e tempos de resposta, permitindo que \glspl{ISP} reconheçam e bloqueiem a conexão, mesmo quando encriptada.

\subsubsection{A Ascensão do WireGuard e Túneis de Alta Performance}

Como resposta às limitações supracitadas, o protocolo WireGuard emerge na literatura analisada como o novo estado da arte para comunicações seguras em dispositivos com recursos limitados. \cite{hohmann_bridge_2021} e \cite{dang_ewdc_2025} fornecem evidência empírica de que o WireGuard introduz uma latência mínima (na ordem dos microsegundos) e um overhead de processamento significativamente inferior aos antecessores. A sua eficiência em hardware \textit{low-cost} (como Raspberry Pi) é validada por \cite{bezenk_remote_2024}, tornando-o ideal para a arquitetura Edge do \gls{FUSE}.
Mais relevante ainda, \cite{resende_sistema_2025} valida integralmente uma stack tecnológica semelhante à proposta para o \gls{FUSE}: utilização de WireGuard num \gls{SBC} para possibilitar que um \gls{NVR} baseado em ZoneMinder aceda de forma segura à câmara. \cite{tong_daiottalk_2025} reforça esta abordagem ao demonstrar que arquiteturas descentralizadas, onde a comunicação ocorre diretamente entre o produtor e o consumidor de dados (P2P), são significativamente mais eficientes do que modelos que dependem de um servidor central para o encaminhamento de todo o tráfego, apresentando ganhos de desempenho na ordem de 3x a 5x.

\subsubsection{Análise Crítica}

A síntese dos estudos permite validar as decisões arquiteturais preliminares do \gls{FUSE}. A literatura suporta a implementação de um ''Secure Gateway'' na Edge \cite{garcia_containerized_2023}, capaz de gerir não apenas o vídeo, mas também o ciclo de vida e atualizações do sistema de forma segura \cite{k_efficient_2024}. Contrariando a intuição de que a encriptação degrada a performance, \cite{avanzato_enhancing_2023} sugere que o encapsulamento em túneis modernos pode, de facto, melhorar a qualidade de experiência em redes instáveis ao gerir melhor a fragmentação e perda de pacotes.

Importa referir que quatro dos estudos selecionados na filtragem \gls{PRISMA} não foram aprofundados na discussão principal por apresentarem redundância conceptual ou limitações de âmbito face aos restantes. \cite{trabelsi_fog_2024} propõe uma \gls{VPN} distribuída baseada em \textit{fog computing} e \textit{blockchain}, uma abordagem que, embora inovadora, introduz complexidade excessiva sem vantagens comprovadas para o cenário específico de vídeo ponto-a-ponto do \gls{FUSE}, já coberto de forma mais eficiente pelas soluções de \gls{P2P} citadas. \cite{hussain_securing_2019} discute a otimização de parâmetros em OpenVPN (AdamVPN), focando-se em melhorar um protocolo \textit{legacy} que o presente projeto opta por substituir integralmente por WireGuard. Similarmente, \cite{qaraqe_publicvision_2024} apresenta um sistema de vigilância seguro, mas recorre a arquiteturas centralizadas clássicas (IPSec/DMVPN) cujas limitações já foram extensamente caracterizadas por outros autores. Por fim, \cite{maceda_scada_2022} valida o uso de OpenVPN para sistemas \textit{legacy} (SCADA), reforçando conclusões já estabelecidas sobre a viabilidade de \glspl{VPN} para controlo, mas sem acrescentar novidade ao debate sobre \textit{streaming} de vídeo de alta performance.

\subsubsection{Conclusão}

A resposta à LRRQ2 aponta inequivocamente para a adoção de protocolos baseados em \gls{UDP} (como Wireguard) e conexões iniciadas na Edge com tecnologia \gls{P2P} para garantir soberania de dados, baixa latência e resiliência em redes não controladas, rejeitando soluções baseadas em Cloud centralizada ou \glspl{VPN} \gls{TCP} \textit{legacy}.

Concretamente, a arquitetura do \gls{FUSE} materializa-se na disponibilização de cada câmara (ou conjunto de câmaras) em conjunto com um \gls{SBC}, especificamente um Raspberry Pi. Este dispositivo atuará como um gateway de rede, controlando o tráfego e estabelecendo o encaminhamento através de uma \gls{VPN} gerida pelo Tailscale. Desta forma, o servidor central consegue agregar e controlar o \textit{streaming} de vídeo de todas as câmaras dispersas, garantindo a segurança sem expor os dispositivos diretamente à rede pública.

\section{Estado da Arte em Visão Computacional}
\label{sec:estado_arte_visao}

Relativamente à terceira sub-questão, a RQ1.3, referente à automatização da análise de vídeo, optou-se por uma abordagem de Revisão Narrativa e Exploratória, como descrita por Maria J. Grant e Andrew Booth em \cite{Grant2009} como ''State-of-the-art review''. Esta modalidade metodológica caracteriza-se pela sua flexibilidade na análise crítica da literatura atual, permitindo identificar conceitos-chave, padrões arquiteturais e soluções técnicas emergentes sem a rigidez protocolar de uma revisão sistemática e formal.

Esta opção justifica-se pela vertiginosa evolução dos modelos de \gls{AI} e pela necessidade de analisar não apenas literatura académica clássica, mas também documentação técnica de modelos recentes e \textit{benchmarks} da indústria. A análise encontra-se segmentada em três domínios fundamentais que correspondem às fases da pipeline de processamento proposta para o \gls{FUSE}:

\begin{enumerate}
    \item Deteção de Movimento (Fase 1)
    \item Deteção de Objetos (Fase 2)
    \item Visual-Language Models - \glspl{VLM} (Fase 3)
\end{enumerate}

Apesar da natureza exploratória, esta pesquisa manterá o rigor científico na seleção de fontes, priorizando publicações de menos de 1 ano, dado o exponencial e recente crescimento tecnológico da área, e repositórios open-source com forte validação comunitária. A pesquisa foi orientada pelas seguintes \textit{keywords}, agrupadas por domínio:

\begin{itemize}
    \item \textbf{Fase 1 (Pré-processamento):} Background Subtraction, Motion Detection Algorithms, Frame Differencing efficiency, Video Activity Detection.
    \item \textbf{Fase 2 (Classificação):} Real-time Object Detection, \gls{YOLO} architecture, One-stage detectors, \gls{CNN} inference optimization, Edge \gls{AI}.
    \item \textbf{Fase 3 (Extração de Atributos):} Vision-Language Models (\glspl{VLM}), Multimodal \gls{AI}, Zero-Shot Learning, Open-vocabulary detection, Visual Question Answering.
\end{itemize}

\subsection{Deteção de Movimento e Filtragem Temporal}
\label{subsec:deteção_movimento}

%A primeira etapa da revisão incidirá sobre métodos de pré-processamento e filtragem de vídeo, essenciais para a eficiência global do sistema. O estudo focará na comparação entre algoritmos clássicos de processamento de imagem (baseados em diferenças de pixéis e estatística de cena) e abordagens mais modernas de ``lightweight \gls{AI}''. O objetivo principal será identificar técnicas capazes de filtrar eficazmente segmentos de vídeo sem atividade relevante, minimizando o uso de recursos computacionais (\gls{CPU}/\gls{GPU}) e garantindo robustez face a mudanças de iluminação ou ruído visual, sem descartar falsos negativos críticos. A literatura evidencia uma divisão clara entre abordagens clássicas de processamento de imagem, que privilegiam a eficiência computacional e simplicidade de implementação, e abordagens baseadas em modelos leves de \gls{AI}, que oferecem maior robustez semântica à custa de maior complexidade. Esta tensão entre eficiência e capacidade de generalização constitui um dos principais trade-offs analisados nesta fase da pipeline.

\subsection{Deteção e Classificação de Objetos (Object Detection)}
\label{subsec:deteção_objetos}

%Neste domínio, a literatura será analisada com foco em arquiteturas de Redes Neuronais Convolucionais (\glspl{CNN}) otimizadas para inferência em tempo real. Será dado destaque preponderante à análise da família de arquiteturas \gls{YOLO} \cite{Jocher2023}, atualmente considerada o padrão de indústria para o equilíbrio entre velocidade e precisão \cite{GOSWAMI2024111921}. O estudo comparativo visará determinar qual a versão ou variação desta arquitetura melhor se adequa aos requisitos forenses do projeto, avaliando métricas como a capacidade de deteção de objetos pequenos ou distantes e o desempenho em hardware com recursos limitados. Os estudos analisados preliminarmente revelam um debate recorrente entre arquiteturas altamente precisas, mas computacionalmente exigentes, e modelos otimizados para inferência em tempo real.

\subsection{Modelos de Visão-Linguagem (\glspl{VLM})}
\label{subsec:vlm}

%Por fim, explora-se a fronteira mais recente da Inteligência Artificial: a integração entre visão computacional e processamento de linguagem natural. A revisão abordará o estado da arte dos \glspl{VLM}, analisando arquiteturas multimodais recentes, como por exemplo a família Qwen-VL, entre outras emergentes. O foco da investigação será compreender como estes modelos permitem a extração de atributos complexos e a realização de pesquisas em linguagem natural, avaliando a sua viabilidade de integração num pipeline local em termos de latência e exigência de memória.

\section{Identificação da Lacuna Literária}
\label{sec:lacuna_literaria}
é Necessário este tópico? ou isto era apenas para complementar o extended abstract?
% Da análise preliminar da literatura resulta uma lacuna clara: embora existam estudos sólidos sobre integração de dispositivos \gls{IoT}, protocolos de comunicação segura e modelos avançados de visão computacional, estes domínios são maioritariamente abordados de forma isolada. Verifica-se a ausência de uma arquitetura integrada que combine, de forma sistemática, a interoperabilidade segura de câmaras \gls{CCTV} localizadas em redes não controladas com pipelines de análise automática de vídeo orientadas a contextos forenses. É precisamente nesta interseção, entre segurança de comunicação, abstração de hardware heterogéneo e inteligência analítica aplicada, que o presente trabalho se posiciona.
