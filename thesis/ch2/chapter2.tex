% Chapter 2

\chapter{Estudo do Estado da Arte}
\label{chap:estado_da_arte}

Este capítulo apresenta a revisão de literatura fundamentada no âmbito do projeto, analisando o estado atual das tecnologias críticas para o desenvolvimento do \gls{FUSE}. O objetivo principal desta revisão é responder às sub-questões operacionais previamente identificadas, nomeadamente a RQ1.1 e a RQ1.2, estabelecendo uma base teórica sólida para as decisões de arquitetura e segurança.

Adicionalmente, e dada a natureza aplicada da RQ1.3, é conduzido um estudo técnico aprofundado sobre os paradigmas atuais de visão computacional. Este estudo visa não apenas identificar o estado da arte, mas também comparar padrões, pipelines de processamento e modelos de \gls{AI} passíveis de integração na plataforma.

\section{Processo de Investigação}
\label{sec:processo_investigacao}

O Processo de investigação foi guiado pelo mapeamento das duas primeiras sub-questões da presente dissertação em questões de pesquisa com foco na revisão literária, conforme descrito na Tabela \ref{tab:questoes_pesquisa}.

\begin{table}[h]
\centering
\caption{Questões de Pesquisa Orientadas à Revisão Literária}
\label{tab:questoes_pesquisa}
\begin{tabular}{|l|l|p{5cm}|p{5cm}|}
\hline
\textbf{RQ ID} & \textbf{LRRQ ID} & \textbf{Description} & \textbf{Tópicos e Keywords de pesquisa} \\ \hline
RQ1.1 & LRRQ1 & Quais são as arquiteturas de referência e padrões de design de software descritos na literatura para a abstração e interoperabilidade de dispositivos \gls{IoT} heterogéneos? & \gls{IoT} Interoperability, Hardware Abstraction Layer, Middleware patterns, \gls{ONVIF} standardization, Heterogeneous device integration. \\ \hline
RQ1.2 & LRRQ2 & Qual o estado da arte em protocolos de comunicação segura e \glspl{VPN} para acesso remoto e streaming? & Secure Tunneling Protocols, \gls{VPN} performance analysis, Streaming encryption, \gls{NAT} Traversal, \gls{ZTNA}. \\ \hline
\end{tabular}
\end{table}

Cada questão de pesquisa será enquadrada de acordo com o modelo \gls{PICOCS} \cite{Alberto2015}, garantindo o rigor na seleção das fontes utilizadas. A informação será depois selecionada seguindo o fluxo \gls{PRISMA} \cite{Moher2010}. O processo inclui a definição de keywords de pesquisa, a aplicação estrita de critérios de inclusão e exclusão, seguida de uma filtragem em etapas e, finalmente, uma discussão crítica sobre a aplicabilidade dos estudos selecionados para a arquitetura do \gls{FUSE}.

Como fator inclusivo de seleção, selecionou-se, por exemplo, a data de publicação posterior a 2018. A escolha deste intervalo visa garantir que as arquiteturas, frameworks e algoritmos analisados representam o atual estado da arte, evitando a adoção de paradigmas que, embora válidos no passado, não refletem as necessidades de desempenho e escalabilidade dos sistemas modernos. Foram privilegiados artigos revistos por pares, normas técnicas e literatura técnica amplamente reconhecida. Excluíram-se estudos puramente teóricos sem aplicação prática ao domínio da videovigilância ou da integração de dispositivos, tal como trabalhos relativos a soluções proprietárias fechadas sem documentação técnica pública suficiente.

\section{Estado da Arte em Visão Computacional}
\label{sec:estado_arte_visao}

Relativamente à terceira sub-questão, a RQ1.3, referente à automatização da análise de vídeo, optou-se por uma abordagem de Revisão Narrativa e Exploratória, como descrita por Maria J. Grant e Andrew Booth em \cite{Grant2009} como ``State-of-the-art review''. Esta modalidade metodológica caracteriza-se pela sua flexibilidade na análise crítica da literatura atual, permitindo identificar conceitos-chave, padrões arquiteturais e soluções técnicas emergentes sem a rigidez protocolar de uma revisão sistemática e formal.

Esta opção justifica-se pela vertiginosa evolução dos modelos de \gls{AI} e pela necessidade de analisar não apenas literatura académica clássica, mas também documentação técnica de modelos recentes e benchmarks da indústria. A análise encontra-se segmentada em três domínios fundamentais que correspondem às fases da pipeline de processamento proposta para o \gls{FUSE}:

\begin{enumerate}
    \item Deteção de Movimento (Fase 1)
    \item Deteção de Objetos (Fase 2)
    \item Visual-Language Models - \glspl{VLM} (Fase 3)
\end{enumerate}

Apesar da natureza exploratória, esta pesquisa manterá o rigor científico na seleção de fontes, priorizando publicações de menos de 1 ano, dado o exponencial e recente crescimento tecnológico da área, e repositórios open-source com forte validação comunitária. A pesquisa foi orientada pelas seguintes keywords, agrupadas por domínio:

\begin{itemize}
    \item \textbf{Fase 1 (Pré-processamento):} Background Subtraction, Motion Detection Algorithms, Frame Differencing efficiency, Video Activity Detection.
    \item \textbf{Fase 2 (Classificação):} Real-time Object Detection, \gls{YOLO} architecture, One-stage detectors, \gls{CNN} inference optimization, Edge \gls{AI}.
    \item \textbf{Fase 3 (Extração de Atributos):} Vision-Language Models (\glspl{VLM}), Multimodal \gls{AI}, Zero-Shot Learning, Open-vocabulary detection, Visual Question Answering.
\end{itemize}

\subsection{Deteção de Movimento e Filtragem Temporal}
\label{subsec:deteção_movimento}

A primeira etapa da revisão incidirá sobre métodos de pré-processamento e filtragem de vídeo, essenciais para a eficiência global do sistema. O estudo focará na comparação entre algoritmos clássicos de processamento de imagem (baseados em diferenças de pixéis e estatística de cena) e abordagens mais modernas de ``lightweight \gls{AI}''. O objetivo principal será identificar técnicas capazes de filtrar eficazmente segmentos de vídeo sem atividade relevante, minimizando o uso de recursos computacionais (\gls{CPU}/\gls{GPU}) e garantindo robustez face a mudanças de iluminação ou ruído visual, sem descartar falsos negativos críticos. A literatura evidencia uma divisão clara entre abordagens clássicas de processamento de imagem, que privilegiam a eficiência computacional e simplicidade de implementação, e abordagens baseadas em modelos leves de \gls{AI}, que oferecem maior robustez semântica à custa de maior complexidade. Esta tensão entre eficiência e capacidade de generalização constitui um dos principais trade-offs analisados nesta fase da pipeline.

\subsection{Deteção e Classificação de Objetos (Object Detection)}
\label{subsec:deteção_objetos}

Neste domínio, a literatura será analisada com foco em arquiteturas de Redes Neuronais Convolucionais (\glspl{CNN}) otimizadas para inferência em tempo real. Será dado destaque preponderante à análise da família de arquiteturas \gls{YOLO} \cite{Jocher2023}, atualmente considerada o padrão de indústria para o equilíbrio entre velocidade e precisão \cite{GOSWAMI2024111921}. O estudo comparativo visará determinar qual a versão ou variação desta arquitetura melhor se adequa aos requisitos forenses do projeto, avaliando métricas como a capacidade de deteção de objetos pequenos ou distantes e o desempenho em hardware com recursos limitados. Os estudos analisados preliminarmente revelam um debate recorrente entre arquiteturas altamente precisas, mas computacionalmente exigentes, e modelos otimizados para inferência em tempo real.

\subsection{Modelos de Visão-Linguagem (Vision-Language Models)}
\label{subsec:vlm}

Por fim, explora-se a fronteira mais recente da Inteligência Artificial: a integração entre visão computacional e processamento de linguagem natural. A revisão abordará o estado da arte dos \glspl{VLM}, analisando arquiteturas multimodais recentes, como por exemplo a família Qwen-VL, entre outras emergentes. O foco da investigação será compreender como estes modelos permitem a extração de atributos complexos e a realização de pesquisas em linguagem natural, avaliando a sua viabilidade de integração num pipeline local em termos de latência e exigência de memória.

\section{Identificação da Lacuna Literária}
\label{sec:lacuna_literaria}

Da análise preliminar da literatura resulta uma lacuna clara: embora existam estudos sólidos sobre integração de dispositivos \gls{IoT}, protocolos de comunicação segura e modelos avançados de visão computacional, estes domínios são maioritariamente abordados de forma isolada. Verifica-se a ausência de uma arquitetura integrada que combine, de forma sistemática, a interoperabilidade segura de câmaras \gls{CCTV} localizadas em redes não controladas com pipelines de análise automática de vídeo orientadas a contextos forenses. É precisamente nesta interseção, entre segurança de comunicação, abstração de hardware heterogéneo e inteligência analítica aplicada, que o presente trabalho se posiciona.
