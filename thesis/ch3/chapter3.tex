% Chapter 3

\chapter{Metodologia}
\label{chap:metodologia}

A metodologia adotada para a realização deste trabalho académico baseia-se no modelo de ``Research Process'' proposto por B. J. Oates no livro ``Researching Information Systems and Computing'' \cite{Oates2006}, ilustrado na Figura \ref{fig:research_process}.

\begin{figure}[H]
\centering
\includegraphics[width=\textwidth]{ch3/assets/research-process.png} % Placeholder para imagem
\caption{Diagrama de ''Research Proces'' de B. J. Oates \cite{Oates2006}, aplicado ao contexto do projeto.}
\label{fig:research_process}
\end{figure}

O ponto de partida da investigação enquadra-se em ``Experiences and motivation'', dado que a génese deste projeto deriva diretamente da observação da ineficiência e limitações técnicas enfrentadas pelos \glspl{OPC} e gestores de segurança na recolha de imagens de \gls{CCTV}. Através de feedback adquirido através de questionários e entrevistas com Núcleos (\gls{GNR}) e Esquadras (\gls{PSP}) de Investigação Criminal (Anexo A), estes problemas são confirmados e registrados. Complementarmente, é realizada uma revisão de literatura preliminar para compreender o estado da arte em protocolos de comunicação e visão computacional. Esta análise permitiu identificar a ausência de soluções integradas que garantam simultaneamente segurança em redes hostis e inteligência analítica avançada. A junção desta necessidade prática (experiência) com a identificação desta lacuna teórica (literatura) resultou na formalização das perguntas de investigação apresentadas anteriormente.

A estratégia central adotada para este trabalho é a de ``Design and Creation''. Esta abordagem é a mais adequada para projetos de Engenharia de Software cujo objetivo primário é o desenvolvimento de um protótipo que visa resolver muitos dos problemas práticos observados e identificados em determinado cenário, como forma de validação de futura implementação ou desenvolvimento de algo mais avançado. Dentro desta estratégia inserem-se quatro principais passos, nomeadamente Design, Desenvolvimento, Testes e Conclusões.

\begin{itemize}
    \item \textbf{Design:} O foco será a proposta de arquitetura a seguir no passo seguinte de desenvolvimento.
    \item \textbf{Desenvolvimento:} Terá como principal objetivo a implementação do protótipo (\gls{MVP}) e da pipeline de análise de vídeo.
    \item \textbf{Testes:} Incluirão uma avaliação da consistência e qualidade do código, tal como implementação em cenário real, que possibilitará a obtenção de resultados.
    \item \textbf{Conclusões:} Interpretação dos resultados obtidos para validação da solução.
\end{itemize}

Relativamente aos resultados e à sua avaliação, será usado o método de ``Interviews'' com utilizadores finais, sendo estes \glspl{OPC} ou encarregados de segurança de uma outra entidade, de modo a recolher feedback e validação do funcionamento do protótipo e respetiva utilidade do mesmo. Em adição, será também usado o método de ``Observation'', onde se poderá obter e quantificar resultados e taxas de correspondência na análise automática de eventos e objetos.

Quanto à análise de dados, optou-se por uma abordagem mista que integra os dois métodos disponíveis: ``Quantitative'' e ``Qualitative''. A análise quantitativa incidirá sobre as métricas técnicas recolhidas durante os testes do protótipo, tais como as taxas de precisão e alerta na deteção e classificação de objetos (Fase 2 e 3 da pipeline), a latência do streaming via túnel \gls{VPN}, os tempos de processamento da análise de vídeo, entre outros. Já a análise qualitativa será aplicada à interpretação do feedback dos utilizadores e das observações de cenário real, avaliando o impacto operacional da ferramenta, a eficácia percebida na redução do tempo de investigação e a adequação da interface aos processos de trabalho dos \glspl{OPC}.
