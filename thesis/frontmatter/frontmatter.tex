% we include the glossary here (frontmatter is included with \input, so this command is as if it was in main.tex)
%All acronyms must be written in this file.
\newacronym{NVR}{NVR}{Network Video Recorder}
\newacronym{LAN}{LAN}{Local Area Network}
\newacronym{LEA}{LEA}{Law Enforcement Agencies}
\newacronym{P2P}{P2P}{Peer-to-Peer}
\newacronym{CCTV}{CCTV}{Closed Circuit Television}
\newacronym{OPC}{OPC}{Órgão de Polícia Criminal}
\newacronym{GNR}{GNR}{Guarda Nacional Republicana}
\newacronym{PSP}{PSP}{Polícia de Segurança Pública}
\newacronym{FUSE}{FUSE}{Flexible Universal Stream Engine}
\newacronym{MVP}{MVP}{Minimum Viable Product}
\newacronym{RTSP}{RTSP}{Real Time Streaming Protocol}
\newacronym{VPN}{VPN}{Virtual Private Network}
\newacronym{ABAC}{ABAC}{Action/Attribute Based Access Control}
\newacronym{IoT}{IoT}{Internet of Things}
\newacronym{ONVIF}{ONVIF}{Open Network Video Interface Forum}
\newacronym{NAT}{NAT}{Network Address Translation}
\newacronym{LRRQ}{LRRQ}{Literature Review Research Question}
\newacronym{VLM}{VLM}{Vision-Language Model}
\newacronym{WBS}{WBS}{Work Breakdown Structure}
\newacronym{PRISMA}{PRISMA}{Preferred Reporting Items for Systematic reviews and Meta-Analysis}
\newacronym{PREPD}{PREPD}{Preparação para Dissertação}
\newacronym{DIMEI}{DIMEI}{Dissertação do Mestrado em Engenharia Informática}
\newacronym{PICOCS}{PICOCS}{Population, Intervention, Comparison, Outcomes, Context, Study}
\newacronym{CNN}{CNN}{Convolutional Neural Network}
\newacronym{YOLO}{YOLO}{You Only Look Once}
\newacronym{ACM}{ACM}{Association for Computing Machinery}
\newacronym{IEEE-CS}{IEEE-CS}{IEEE Computer Society}
\newacronym{GPU}{GPU}{Graphics Processing Unit}
\newacronym{CPU}{CPU}{Central Processing Unit}
\newacronym{AI}{AI}{Artificial Intelligence}
\newacronym{RTS}{RTS}{Real-Time System}
\newacronym{GPOS}{GPOS}{General Purpose Operating System}
\newacronym{RTOS}{RTOS}{Real-Time Operating System}
\newacronym{PGF}{PGF}{Portable Graphics Format}
\newacronym{ZTNA}{ZTNA}{Zero Trust Network Access}


\frontmatter % Use roman page numbering style (i, ii, iii, iv...) for the pre-content pages

\pagestyle{plain} % Default to the plain heading style until the thesis style is called for the body content

%----------------------------------------------------------------------------------------
%	TITLE PAGE
%----------------------------------------------------------------------------------------

\maketitlepage


%----------------------------------------------------------------------------------------
%	STATEMENT of INTEGRITY
%----------------------------------------------------------------------------------------
\integritystatement

%----------------------------------------------------------------------------------------
%	DEDICATION  (optional)
%----------------------------------------------------------------------------------------
%
%\dedicatory{For/Dedicated to/To my\ldots}
\begin{dedicatory}
The dedicatory is optional. Below is an example of a humorous dedication.

"To my wife Marganit and my children Ella Rose and Daniel Adam without whom this book would have been completed two years earlier." in "An Introduction To Algebraic Topology" by Joseph J. Rotman.
\end{dedicatory}

%----------------------------------------------------------------------------------------
%	ABSTRACT PAGE
%----------------------------------------------------------------------------------------

\begin{abstract}

% here you put the abstract in the main language of the work.

O presente trabalho de dissertação propõe o desenvolvimento do \gls{FUSE}, uma plataforma de software inovadora concebida para centralizar e organizar sistemas de videovigilância heterogéneos, geograficamente dispersos e situados em redes não controladas. Esta investigação surge em resposta a desafios críticos identificados na segurança pública e privada, nomeadamente a fragmentação tecnológica, as vulnerabilidades de segurança inerentes às comunicações \gls{P2P} e a ineficiência operacional enfrentada por entidades como os \glspl{OPC} na análise forense de vídeo.

A arquitetura proposta integra uma camada de abstração de hardware que normaliza a comunicação com câmaras de diversos fabricantes através do protocolo \gls{RTSP}, garantindo a interoperabilidade. A segurança das comunicações é assegurada através de túneis \gls{VPN} que protegem a integridade e confidencialidade dos dados em redes não controladas, complementada por um modelo de controlo de acessos \gls{ABAC}. Adicionalmente, o sistema incorpora um módulo de visão computacional progressivo, estruturado em três fases: deteção de atividade para filtragem temporal, classificação de objetos (e.g., humanos, veículos) e extração de atributos específicos, visando a automatização da análise forense e a redução significativa da carga de trabalho manual.

A metodologia adotada segue o paradigma de "Design and Creation", com a validação da solução a ser realizada através de um protótipo funcional (\gls{MVP}) em cenários simulados e reais. Os resultados esperados centram-se na demonstração da eficácia da integração segura de dispositivos, na robustez da proteção de dados e na otimização dos processos de investigação criminal, sempre em estrita observância dos princípios éticos e regulamentares, como o EU AI Act.


\end{abstract}

\begin{abstractotherlanguage}
% here you put the abstract in the "other language": English, if the work is written in Portuguese; Portuguese, if the work is written in English.

This dissertation presents the development of \gls{FUSE}, a software platform designed to unify and secure heterogeneous video surveillance systems dispersed across uncontrolled networks. Addressing the challenges of technological fragmentation and security vulnerabilities in current \gls{CCTV} infrastructures, \gls{FUSE} provides a centralized solution for entities such as \glspl{LEA}

The proposed architecture features a hardware abstraction layer for normalizing camera communications via \gls{RTSP} and secures data transmission through VPN tunneling. It implements \gls{ABAC} as an access control method, and integrates a three-stage computer vision pipeline for automated activity detection, object classification, and attribute extraction. This approach aims to enhance interoperability, data security, and forensic analysis efficiency, adhering to ethical standards like the EU AI Act.


\end{abstractotherlanguage}

%----------------------------------------------------------------------------------------
%	ACKNOWLEDGEMENTS (optional)
%----------------------------------------------------------------------------------------

\begin{acknowledgements}

The optional Acknowledgment goes here\ldots Below is an example of a humorous acknowledgment.

"I'd also like to thank the Van Allen belts for protecting us from the harmful solar wind, and the earth for being just the right distance from the sun for being conducive to life, and for the ability for water atoms to clump so efficiently, for pretty much the same reason. Finally, I'd like to thank every single one of my forebears for surviving long enough in this hostile world to procreate. Without any one of you, this book would not have been possible." in "The Woman Who Died a Lot" by Jasper Fforde.
\end{acknowledgements}

%----------------------------------------------------------------------------------------
%	LIST OF CONTENTS/FIGURES/TABLES PAGES
%----------------------------------------------------------------------------------------

\tableofcontents % Prints the main table of contents

\listoffigures % Prints the list of figures

\listoftables % Prints the list of tables

\iflanguage{portuguese}{
\renewcommand{\listalgorithmname}{Lista de Algor\'itmos}
}
\listofalgorithms % Prints the list of algorithms
\addchaptertocentry{\listalgorithmname}


\renewcommand{\lstlistlistingname}{List of Source Code}
\iflanguage{portuguese}{
\renewcommand{\lstlistlistingname}{Lista de C\'odigo}
}
\lstlistoflistings % Prints the list of listings (programming language source code)
\addchaptertocentry{\lstlistlistingname}


%----------------------------------------------------------------------------------------
%	ABBREVIATIONS
%----------------------------------------------------------------------------------------
%\begin{abbreviations}{ll} % Include a list of abbreviations (a table of two columns)
%%\textbf{LAH} & \textbf{L}ist \textbf{A}bbreviations \textbf{H}ere\\
%%\textbf{WSF} & \textbf{W}hat (it) \textbf{S}tands \textbf{F}or\\
%\end{abbreviations}

%----------------------------------------------------------------------------------------
%	SYMBOLS
%----------------------------------------------------------------------------------------

%\begin{symbols}{lll} % Include a list of Symbols (a three column table)
%
%$a$ & distance & \si{\meter} \\
%$P$ & power & \si{\watt} (\si{\joule\per\second}) \\
%%Symbol & Name & Unit \\
%
%\addlinespace % Gap to separate the Roman symbols from the Greek
%
%$\omega$ & angular frequency & \si{\radian} \\
%
%\end{symbols}



%----------------------------------------------------------------------------------------
%	ACRONYMS
%----------------------------------------------------------------------------------------

\newcommand{\listacronymname}{List of Acronyms}
\iflanguage{portuguese}{
\renewcommand{\listacronymname}{Lista de Acr\'onimos}
}

%Use GLS
\glsresetall
\printglossary[title=\listacronymname,type=\acronymtype,style=long]

%----------------------------------------------------------------------------------------
%	DONE
%----------------------------------------------------------------------------------------

\mainmatter % Begin numeric (1,2,3...) page numbering
\pagestyle{thesis} % Return the page headers back to the "thesis" style
