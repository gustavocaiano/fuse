% Chapter 5

\chapter{Ética}
\label{chap:etica}

A realização deste trabalho rege-se pelos princípios de integridade, responsabilidade e rigor científico descritos no Código de Boas Práticas e de Conduta do P.PORTO \cite{PPORTO2020}.

Do ponto de vista profissional, o trabalho enquadra-se nos princípios estruturantes da engenharia de software, alinhados com referências internacionais como o Software Engineering Code of Ethics and Professional Practice, desenvolvido em conjunto pela \gls{ACM} e pela \gls{IEEE-CS} \cite{ACM_Ethics}. A conformidade com estes princípios garante a qualidade e segurança dos sistemas desenvolvidos, a responsabilidade perante clientes e utilizadores, e a honestidade na comunicação de resultados, limitações e riscos associados às soluções tecnológicas.

O \gls{FUSE} envolve uma pipeline de análise automática aplicada a contextos de videovigilância, potencialmente incidindo sobre a obtenção de dados pessoais identificáveis. Estes dados sensíveis apenas serão obtidos num contexto de validação do \gls{MVP} em cenário real, que estará sempre salvaguardado por requerimento de despacho judicial para fornecimento de meios técnicos e respetiva autorização de captação de imagem. Ainda assim, a pipeline é desenhada em estrita observância do EU AI Act \cite{EU_AI_Act}, garantindo que o sistema não incorre em ``Unacceptable risks''. O sistema não se destina, nem permite, práticas proibidas tais como:

\begin{itemize}
    \item Social scoring ou manipulação comportamental;
    \item Recolha indiscriminada (untargeted scraping) de imagens de \gls{CCTV} para criação de bases de dados de reconhecimento facial;
    \item Identificação biométrica remota em tempo real em espaços públicos para fins policiais (real-time remote biometric identification).
\end{itemize}

No âmbito do último ponto, apesar do alvo ser legislar o reconhecimento biométrico em tempo-real, justifica-se a conformidade pela natureza da análise automática, que será realizada sob gravações dos eventos, e não sob o próprio streaming.

A análise focada na ``Fase 3'' da pipeline de deteção automática restringe-se à classificação de características objetivas (e.g., cor de vestuário, tipo de veículo) para auxiliar a pesquisa forense, mantendo sempre o princípio da validação humana, onde a decisão final cabe ao agente humano e não ao algoritmo.

