% Chapter 5

\chapter{Considerações Finais}
\label{chap:conclusoes}
O presente capítulo tem como objetivo sintetizar os resultados obtidos durante a fase de preparação da dissertação, consolidando as decisões tomadas com base na investigação realizada. Apresentam-se, de seguida, as principais conclusões retiradas do estudo do estado da arte e do planeamento, bem como a rota detalhada para o desenvolvimento futuro do projeto \gls{FUSE}.

\section{Conclusões}
\label{sec:conclusoes_finais}

O estudo do estado da arte, detalhado no \autoref{chap:estado_da_arte}, foi determinante para as decisões arquiteturais do projeto. A análise literária confirmou que a interoperabilidade entre dispositivos heterogéneos exige uma camada de abstração robusta baseada em microserviços. No domínio da segurança, a investigação apontou para a obsolescência das \glspl{VPN} tradicionais em cenários de \textit{streaming} de vídeo, validando a adoção de túneis modernos como o WireGuard e topologias \gls{P2P} para garantir a conectividade segura sem comprometer a performance. 

Adicionalmente, a revisão sobre visão computacional demonstrou a inviabilidade atual do processamento complexo em dispositivos Edge de baixo custo, fundamentando a opção por uma arquitetura de inteligência centralizada que tira partido de modelos avançados como o \gls{YOLO} e \glspl{VLM} para uma análise semântica aprofundada.

O planeamento apresentado no \autoref{chap:planeamento}, suportado por uma análise de riscos e cronograma detalhado, demonstra a possibilidade da execução técnica e temporal do projeto, estabelecendo uma rota clara para a fase de implementação que se segue.

\section{Trabalho Futuro}
\label{sec:trabalho_futuro}

Com a conclusão da fase de preparação (\gls{PREPD}), o trabalho transita agora para a execução técnica no âmbito da unidade curricular de dissertação (\gls{DIMEI}). Os passos imediatos e futuros, alinhados com o plano de trabalhos estabelecido, focam-se na materialização do \gls{FUSE} e compreendem:

\begin{itemize}
    \item \textbf{Design Arquitetural:} Formalização da arquitetura do sistema através do modelo de Vistas 4+1, detalhando os componentes e as suas interações.
    \item \textbf{Implementação do MVP:} Desenvolvimento iterativo dos módulos críticos, começando pela infraestrutura de comunicações seguras, seguido pela camada de abstração de câmaras e, finalmente, a \textit{pipeline} de análise de vídeo.
    \item \textbf{Validação em Cenário Real:} Instalação e teste do protótipo em ambiente operacional, permitindo recolher métricas de desempenho e feedback qualitativo dos utilizadores finais.
    \item \textbf{Escrita da Dissertação:} Documentação contínua do processo de desenvolvimento e resultados, culminando na produção do documento final de dissertação que reportará as conclusões do projeto.
\end{itemize}
