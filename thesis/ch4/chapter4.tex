% Chapter 4

\chapter{Planeamento de Trabalho}
\label{chap:planeamento}

Este capítulo descreve o planeamento deste projeto, que foi estruturado para garantir a exequibilidade dos objetivos propostos dentro do prazo académico estipulado. A organização das atividades divide-se entre a fase de preparação (unidade curricular de \gls{PREPD}) e a fase de execução e escrita da dissertação (unidade curricular de \gls{DIMEI}).

\section{Definição do Âmbito e Entregáveis}
\label{sec:ambito_entregaveis}

O âmbito deste projeto foi decomposto através de um \gls{WBS}. Pelo mesmo, ilustrado na Figura \ref{fig:wbs}, estão organizados os entregáveis previstos por cinco fases principais, sendo estas:

\begin{enumerate}
    \item \textbf{Planeamento e Análise:} Focada na definição do problema e estado da arte. Inclui entregáveis como o Project Charter, o próprio \gls{WBS}, um Gantt Chart, um questionário e respetivas respostas dos \glspl{OPC}, este Extended Abstract e a revisão da literatura incluída no mesmo.
    \item \textbf{Design:} Dedicada à modelação da solução. Os principais entregáveis previstos são a documentação da Arquitetura de Software (Vistas 4+1) e o Modelo de Domínio, assegurando que a estrutura do \gls{FUSE} é robusta antes da implementação.
    \item \textbf{Desenvolvimento:} Fase central do projeto, onde será implementado o código da aplicação. Divide-se em três módulos críticos: Comunicações Seguras, Camada de Abstração e Módulo de Análise de Vídeo.
    \item \textbf{Testes:} Validação da solução através de testes unitários e funcionais, em conjunto com a validação do protótipo em ambiente controlado ou cenário real.
    \item \textbf{Conclusões:} Considerações finais do projeto, incluindo a redação final da dissertação, a interpretação dos resultados e a apresentação final.
\end{enumerate}

\begin{figure}[H]
\centering
\includegraphics[width=\textwidth]{ch4/assets/WBS.png} % Placeholder
\caption{WBS do projeto.}
\label{fig:wbs}
\end{figure}

\section{Plano de Trabalho}
\label{sec:plano_trabalho}

O objetivo desta secção é ilustrar e descrever a calendarização das entregas dos respetivos entregáveis mencionados anteriormente, garantindo um acompanhamento rigoroso do progresso do projeto. O planeamento temporal encontra-se representado no diagrama de Gantt, Figura \ref{fig:gantt}.

\begin{figure}[H]
\centering
\includegraphics[width=\textwidth]{ch4/assets/gantt-labeled.png} % Placeholder
\caption{Gantt Chart do projeto.}
\label{fig:gantt}
\end{figure}

A execução do cronograma inicia-se com um primeiro marco de controlo a 15 de dezembro de 2025. O encerramento da fase de planeamento (\gls{PREPD}) está estipulado para 22 de janeiro de 2026. O segundo grande bloco temporal, referente à unidade curricular de \gls{DIMEI}, tem como data-alvo de entrega final o dia 22 de junho de 2026.

O diagrama de Gantt ilustra a distribuição temporal das atividades ao longo de aproximadamente 220 dias úteis, desde o início do projeto em outubro de 2025 até agosto de 2026. A fase de \textbf{Planeamento e Análise}, com duração prevista de 65 dias, concentra-se nos primeiros meses do projeto e inclui a elaboração do Extended Abstract, do Project Charter, do \gls{WBS}, bem como a realização do estudo do estado da arte e a aplicação de questionários junto dos \glspl{OPC}. Esta fase culmina com a produção do próprio Gantt Chart, que depende da conclusão do \gls{WBS}.

A fase de \textbf{Design}, com duração de 17 dias, inicia-se imediatamente após o término do planeamento, em janeiro de 2026. Esta fase contempla a documentação da Arquitetura de Software através das Vistas 4+1 e a modelação do Domínio, estabelecendo as bases arquiteturais para o desenvolvimento subsequente.

O \textbf{Desenvolvimento}, a fase mais extensa com 85 dias de duração, estende-se de fevereiro a junho de 2026. Esta fase estrutura-se em três componentes principais que se desenvolvem de forma parcialmente paralela: as Comunicações Seguras (30 dias), a Camada de Abstração de Câmaras (30 dias) e o Módulo de Análise Automática de Vídeo (45 dias). O módulo de análise de vídeo, por sua vez, divide-se em duas etapas sequenciais: a implementação de uma pipeline bi-fásica (20 dias) seguida da evolução para uma pipeline tri-fásica (25 dias), permitindo uma abordagem incremental na integração das funcionalidades de deteção e classificação.

A fase de \textbf{Testes} inicia-se em paralelo com o desenvolvimento, em abril de 2026, e prolonga-se até ao final do projeto. Esta fase compreende a execução de testes unitários e funcionais durante 43 dias, seguida de um período de validação do protótipo em ambiente real que se estende por 60 dias, permitindo a recolha de feedback dos utilizadores finais e a produção do respetivo relatório.

Como se observa na Figura \ref{fig:gantt}, após a data de entrega da unidade curricular de \gls{DIMEI} a 22 de junho de 2026, ainda existe trabalho planeado que se estende até final de agosto de 2026. Esta extensão temporal deve-se ao facto de a validação do protótipo estar desde já expectada a requerer um período mais alargado do que o estipulado para o término formal da dissertação. Embora o início da fase de validação esteja previsto dentro do prazo académico, a sua conclusão completa poderá estender-se para além da data de entrega da dissertação. O feedback que for possível adquirir durante o período de validação será anexado ao relatório final, permitindo documentar os resultados obtidos e as observações recolhidas junto dos utilizadores finais, mesmo que o processo de validação se prolongue para além do término formal do trabalho académico.

\section{Project Charter e Gestão de Riscos}
\label{sec:gestao_riscos}

O Project Charter constituiu o artefacto inicial deste projeto, servindo como primeiro documento para a formalização da proposta de dissertação. A Tabela \ref{tab:stakeholders} apresenta o mapeamento dos Stakeholders, classificados quanto ao seu poder de influência e nível de interesse no sucesso do projeto, utilizando como base a ``Mendelow's Matrix'' \cite{Mendelow1981}.

\begin{table}[H]
\centering
\caption{Stakeholders identificados}
\label{tab:stakeholders}
\begin{tabular}{lllp{6cm}}
\toprule
\textbf{Nome} & \textbf{Poder} & \textbf{Interesse} & \textbf{Justificação} \\
\midrule
StabilityBubble & Alto & Alto & Entidade promotora e futura fornecedora/exploradora comercial da solução \gls{FUSE}. \\
\glspl{OPC} & Alto & Alto & Utilizadores finais críticos; o seu feedback valida a utilidade operacional e requisitos forenses. \\
Paulo Baltarejo Sousa & Alto & Médio & Orientação académica e validação científica da metodologia e resultados. \\
Francisco Loureiro & Alto & Alto & Supervisão empresarial e alinhamento do produto com a estratégia de mercado. \\
Gestores de Segurança & Médio & Médio & Potenciais clientes empresariais que beneficiam da centralização de sistemas \gls{CCTV}. \\
\bottomrule
\end{tabular}
\end{table}

Relativamente à gestão de riscos, a Tabela \ref{tab:riscos} consolida os riscos identificados, atribuindo-lhes uma probabilidade e um impacto seguindo uma escala de Likert \cite{Likert1932} de 1 a 5.

\begin{table}[H]
\centering
\caption{Identificação e gestão de riscos associados ao projeto}
\label{tab:riscos}
\begin{tabular}{lp{4cm}cccp{5cm}}
\toprule
\textbf{ID} & \textbf{Descrição} & \textbf{P} & \textbf{I} & \textbf{P*I} & \textbf{Resposta} \\
\midrule
1 & Indisponibilidade de Hardware (\gls{GPU}) & 3 & 4 & 12 & Mitigar: Garantir integração modular e desacoplada. \\
2 & Incompatibilidade de Protocolos & 2 & 2 & 4 & Aceitar: Focar o \gls{MVP} na compatibilidade estrita com \gls{RTSP}. \\
3 & Falsos Positivos na Deteção & 3 & 3 & 9 & Mitigar: Implementar sistema de ajuste de sensibilidade configurável. \\
4 & Atraso no Cronograma & 3 & 5 & 15 & Mitigar: Adotar abordagem de desenvolvimento incremental. \\
\bottomrule
\end{tabular}
\end{table}

Pela análise à Tabela \ref{tab:riscos}, destaca-se os riscos 1 e 4 como os mais elevados. A estratégia de mitigação assenta fundamentalmente na modularidade e no desenvolvimento incremental.

