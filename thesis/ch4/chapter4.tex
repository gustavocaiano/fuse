% Chapter 4

\chapter{Planeamento de Trabalho}
\label{chap:planeamento}

Este capítulo descreve o planeamento deste projeto, que foi estruturado para garantir a exequibilidade dos objetivos propostos dentro do prazo académico estipulado. A organização das atividades divide-se entre a fase de preparação (unidade curricular de \gls{PREPD}) e a fase de execução e escrita da dissertação (unidade curricular de \gls{DIMEI}).

\section{Definição do Âmbito e Entregáveis}
\label{sec:ambito_entregaveis}

O âmbito deste projeto foi decomposto através de um \gls{WBS}. Pelo mesmo, ilustrado na Figura \ref{fig:wbs}, estão organizados os entregáveis previstos por cinco fases principais, sendo estas:

\begin{enumerate}
    \item \textbf{Planeamento e Análise:} Focada na definição do problema e estado da arte. Inclui entregáveis como o Project Charter, o próprio \gls{WBS}, um Gantt Chart, um questionário e respetivas respostas dos \glspl{OPC}, este Extended Abstract e a revisão da literatura incluída no mesmo.
    \item \textbf{Design:} Dedicada à modelação da solução. Os principais entregáveis previstos são a documentação da Arquitetura de Software (Vistas 4+1) e o Modelo de Domínio, assegurando que a estrutura do \gls{FUSE} é robusta antes da implementação.
    \item \textbf{Desenvolvimento:} Fase central do projeto, onde será implementado o código da aplicação. Divide-se em três módulos críticos: Comunicações Seguras, Camada de Abstração e Módulo de Análise de Vídeo.
    \item \textbf{Testes:} Validação da solução através de testes unitários e funcionais, em conjunto com a validação do protótipo em ambiente controlado ou cenário real.
    \item \textbf{Conclusões:} Considerações finais do projeto, incluindo a redação final da dissertação, a interpretação dos resultados e a apresentação final.
\end{enumerate}

\begin{figure}[H]
\centering
\includegraphics[width=\textwidth]{ch4/assets/WBS.png} % Placeholder
\caption{WBS do projeto.}
\label{fig:wbs}
\end{figure}

\section{Plano de Trabalho}
\label{sec:plano_trabalho}

O objetivo desta secção é ilustrar e descrever a calendarização das entregas dos respetivos entregáveis mencionados anteriormente, garantindo um acompanhamento rigoroso do progresso do projeto. O planeamento temporal encontra-se representado no diagrama de Gantt, Figura \ref{fig:gantt}.

A execução do cronograma inicia-se com um primeiro marco de controlo a 15 de dezembro de 2025. O encerramento da fase de planeamento (\gls{PREPD}) está estipulado para 22 de janeiro de 2026. O segundo grande bloco temporal, referente à unidade curricular de \gls{DIMEI}, tem como data-alvo de entrega final o dia 22 de junho de 2026.

\begin{figure}[H]
\centering
\includegraphics[width=\textwidth]{ch4/assets/gantt-labeled.png} % Placeholder
\caption{Gantt Chart do projeto.}
\label{fig:gantt}
\end{figure}



O diagrama de Gantt ilustra a distribuição temporal das atividades ao longo de aproximadamente 220 dias úteis, desde o início do projeto em outubro de 2025 até agosto de 2026. A fase de \textbf{Planeamento e Análise}, com duração prevista de 65 dias, concentra-se nos primeiros meses do projeto e inclui a elaboração do Extended Abstract, do Project Charter, do \gls{WBS}, bem como a realização do estudo do estado da arte e a aplicação de questionários junto dos \glspl{OPC}. Esta fase culmina com a produção do próprio Gantt Chart, que depende da conclusão do \gls{WBS}.

A fase de \textbf{Design}, com duração de 17 dias, inicia-se imediatamente após o término do planeamento, em janeiro de 2026. Esta fase contempla a documentação da Arquitetura de Software através das Vistas 4+1 e a modelação do Domínio, estabelecendo as bases arquiteturais para o desenvolvimento subsequente.

O \textbf{Desenvolvimento}, a fase mais extensa com 85 dias de duração, estende-se de fevereiro a junho de 2026. Esta fase estrutura-se em três componentes principais que se desenvolvem de forma parcialmente paralela: as Comunicações Seguras (30 dias), a Camada de Abstração de Câmaras (30 dias) e o Módulo de Análise Automática de Vídeo (45 dias). O módulo de análise de vídeo, por sua vez, divide-se em duas etapas sequenciais: a implementação de uma pipeline bi-fásica (20 dias) seguida da evolução para uma pipeline tri-fásica (25 dias), permitindo uma abordagem incremental na integração das funcionalidades de deteção e classificação.

A fase de \textbf{Testes} inicia-se em paralelo com o desenvolvimento, em abril de 2026, e prolonga-se até ao final do projeto. Esta fase compreende a execução de testes unitários e funcionais durante 43 dias, seguida de um período de validação do protótipo em ambiente real que se estende por 60 dias, permitindo a recolha de feedback dos utilizadores finais e a produção do respetivo relatório.

Como se observa na Figura \ref{fig:gantt}, após a data de entrega da unidade curricular de \gls{DIMEI} a 22 de junho de 2026, ainda existe trabalho planeado que se estende até final de agosto de 2026. Esta extensão temporal deve-se ao facto de a validação do protótipo estar desde já expectada a requerer um período mais alargado do que o estipulado para o término formal da dissertação. Embora o início da fase de validação esteja previsto dentro do prazo académico, a sua conclusão completa poderá estender-se para além da data de entrega da dissertação. O feedback que for possível adquirir durante o período de validação será anexado ao relatório final, permitindo documentar os resultados obtidos e as observações recolhidas junto dos utilizadores finais, mesmo que o processo de validação se prolongue para além do término formal do trabalho académico.

\section{Project Charter e Gestão de Riscos}
\label{sec:gestao_riscos}

O Project Charter (Apêndice \ref{AppendixB}) constituiu o artefacto inicial deste projeto, servindo como primeiro documento para a formalização da proposta de dissertação junto da empresa acolhedora e da instituição académica. Dada a sua natureza preliminar, este documento apresentou uma visão inicial do problema e dos objetivos que, embora alinhados com a missão do \gls{FUSE}, se revelaram amplos, tendo sido posteriormente refinados e concretizados no \gls{WBS} e no plano de trabalhos detalhado anteriormente. As datas e entregáveis originais constantes no Charter foram, por conseguinte, ajustados para refletir um mais correto e detalhado planeamento.

\subsection{Identificação de Stakeholders}
\label{subsec:stakeholder_identification}

Apesar dos ajustes realizados, o Project Charter mantém-se como a referência central para a identificação das partes interessadas (Stakeholders). A Tabela \ref{tab:stakeholders} apresenta o mapeamento atualizado dos Stakeholders, classificando-os quanto ao seu poder de influência e nível de interesse no sucesso do projeto. Como principais diferenças entre a identificação do presente documento e a do Project Charok ter,  destaca-se a inclusão da StabilityBubble, cuja posição estratégica evolui para fornecedor potencial do software pós-desenvolvimento, e a exclusão de entusiastas de Home Automation, inicialmente projetados como possível público-alvo. Esta exclusão dá-se devido à grande apropriação do \gls{FUSE} a entidades e \glspl{OPC}, dado que se torna uma solução demasiado desenvolvida para projetos caseiros e pequenos.

Para a classificação dos Stakeholders, utilizou-se como base a ''Mendelow's Matrix'' \cite{Mendelow1981}. A atribuição dos níveis ('Alto', 'Médio', 'Baixo') obedeceu aos seguintes critérios:

\begin{itemize}
    \item \textbf{Poder:} Capacidade da entidade em influenciar decisões, alocar recursos ou bloquear o andamento do projeto.
    \item \textbf{Interesse:} Grau de impacto que os resultados do projeto terão nas operações ou estratégia da entidade.
\end{itemize}

\begin{table}[H]
\centering
\caption{Identificação de Stakeholders com atribuição de Poder e Interesse}
\label{tab:stakeholders}
\begin{tabular}{lllp{6cm}}
\toprule
\textbf{Nome} & \textbf{Poder} & \textbf{Interesse} & \textbf{Justificação} \\
\midrule
StabilityBubble / NovaForensic & Alto & Alto & Entidade promotora e futura fornecedora/exploradora comercial da solução \gls{FUSE}. \\
\glspl{OPC} & Alto & Alto & Utilizadores finais críticos; o seu feedback valida a utilidade operacional e requisitos forenses. \\
Paulo Baltarejo Sousa (Orientador) & Alto & Médio & Orientação académica e validação científica da metodologia e resultados. \\
Francisco Loureiro (Supervisor) & Alto & Alto & Supervisão empresarial e alinhamento do produto com a estratégia de mercado. \\
Gestores de Segurança & Médio & Médio & Potenciais clientes empresariais que beneficiam da centralização de sistemas \gls{CCTV}. \\
Fornecedores de Câmaras & Baixo & Baixo & O \gls{FUSE} visa a eliminação da dependência de um só fornecedor, que tanto pode ser um novo fator decisivo para a escolha do fabricante. \\
\bottomrule
\end{tabular}
\end{table}

\subsection{Gestão de Riscos}
\label{subsec:risk_management}

Relativamente à gestão de riscos, trata-se de um processo contínuo que se iniciou com o levantamento preliminar no Project Charter. Embora a análise inicial focasse riscos mais genéricos (e.g., dependência de hardware), o planeamento aprofundado permitiu identificar ameaças mais específicas e definir estratégias de mitigação concretas. A Tabela \ref{tab:riscos} consolida os riscos, atribuindo-lhes uma probabilidade e um impacto, combinando alguns identificados na fase inicial com novos riscos técnicos decorrentes da complexidade da análise de vídeo e integração de redes. A quantificação do risco segue uma matriz de probabilidade e impacto (P×I), onde ambas as variáveis são classificadas numa escala de Likert \cite{Likert1932} de 1 a 5. Esta escala foi definida da seguinte forma:

\begin{itemize}
    \item \textbf{Probabilidade (P):} 1 representando um acontecimento raro ou extremamente improvável, e 5 algo que seja extremamente provável acontecer, possivelmente previsto.
    \item \textbf{Impacto (I):} 1 compreende uma insignificância ou ligeiro atraso, enquanto que 5 impacta criticamente, possivelmente impossibilitando o projeto ou condicionando o resultado do mesmo.
\end{itemize}

A multiplicação destes fatores resulta no Risco Quantificado, permitindo priorizar as estratégias de mitigação para as ameaças com pontuação mais elevada, conforme demonstrado na Tabela \ref{tab:riscos}.

\begin{table}[H]
\centering
\caption{Identificação e gestão de riscos associados ao projeto}
\label{tab:riscos}
\begin{tabular}{lp{3cm}p{4cm}cccp{2.5cm}}
\toprule
\textbf{ID} & \textbf{Descrição} & \textbf{Causa} & \textbf{P} & \textbf{I} & \textbf{P*I} & \textbf{Estratégia} \\
\midrule
1 & Indisponibilidade de Hardware (\gls{GPU}) & A análise de vídeo requer hardware gráfico potente, que é escasso ou não disponível & 3 & 4 & 12 & Mitigar \\
2 & Incompatibilidade de Protocolos & Determinada câmara não suporta o protocolo \gls{RTSP}, impossibilitando a integração prevista. & 2 & 2 & 4 & Aceitar \\
3 & Falsos Positivos na Deteção & \textit{Pipeline} de \gls{AI} gera demasiados alertas falsos em condições adversas (chuva, noite). & 3 & 3 & 9 & Mitigar \\
4 & Atraso no Cronograma & A complexidade da integração de múltiplos sistemas excede o tempo previsto para o semestre letivo. & 3 & 5 & 15 & Mitigar \\
\bottomrule
\end{tabular}
\end{table}

A Tabela \ref{tab:riscos} apresenta quatro riscos identificados, dos quais três requerem estratégias de mitigação e um é aceite como parte do âmbito do projeto. Pela análise à tabela, destaca-se os 1º e 4º riscos, Indisponibilidade de Hardware (\gls{GPU}) e Atraso no Cronograma, com os índices de risco mais elevados sendo eles 12 e 15, respetivamente. Por consequente, conclui-se que grande parte do sucesso do projeto e dos resultados obtidos dependerão intrinsecamente da gestão rigorosa dos recursos computacionais e do cumprimento dos prazos estipulados, bem como o seguimento do planeamento efetuado.

As estratégias de resposta detalhadas para cada risco são as seguintes:

\textbf{Risco 1 - Indisponibilidade de Hardware (\gls{GPU}):} A estratégia de mitigação consiste em garantir que a integração com a \textit{pipeline} de análise automática seja feita de forma modular e desacoplada, permitindo a flexibilidade da escolha do modelo de \gls{AI} utilizado conforme o hardware disponível no momento. Esta abordagem assegura que o sistema possa adaptar-se a diferentes configurações de hardware sem comprometer a funcionalidade core.

\textbf{Risco 2 - Incompatibilidade de Protocolos:} A opção pela estratégia de ``Aceitação'' serve para delimitar claramente o âmbito do projeto. O \gls{MVP} a ser desenvolvido não pretende ser universalmente compatível com todo o conjunto de câmaras existentes, mas sim provar o conceito através de um protocolo standard habitualmente presente (\gls{RTSP}), excluindo câmaras que não suportem este protocolo. Esta decisão permite focar o esforço de engenharia na inteligência do sistema e não na compatibilidade de drivers.

\textbf{Risco 3 - Falsos Positivos na Deteção:} A mitigação será realizada através da implementação de um sistema de ajuste de sensibilidade configurável pelo utilizador, permitindo que este defina parâmetros como os segundos necessários de deteção do objeto antes do alerta ser gerado. Esta funcionalidade permite reduzir falsos positivos em condições adversas, como chuva ou condições de pouca luz.

\textbf{Risco 4 - Atraso no Cronograma:} A estratégia de mitigação assenta na adoção de uma abordagem de desenvolvimento incremental. Será assegurada a estabilidade do núcleo de visualização e deteção básica (\textit{pipeline} bi-fásica) como um \gls{MVP} robusto, antes de iterar para as funcionalidades avançadas de extração de atributos (\textit{pipeline} tri-fásica). Caso o desenvolvimento da \textit{pipeline} tri-fásica se revele incompatível com o tempo disponível, esta será tratada como trabalho futuro, garantindo sempre a entrega de um produto funcional.

A estratégia de mitigação para os riscos prioritários (Riscos 1 e 4) assenta fundamentalmente na modularidade e no desenvolvimento incremental. Ao garantir que a arquitetura do sistema é flexível quanto aos modelos de \gls{AI} e que as funcionalidades são entregues por fases, assegura-se que, mesmo na eventualidade destes riscos se materializarem, o projeto resultará sempre num produto funcional (\gls{MVP}), salvaguardando a dissertação. A monitorização destes riscos será realizada periodicamente em cada reunião de ponto de situação com o orientador e supervisor, permitindo o ajuste dinâmico das estratégias de resposta caso a probabilidade ou impacto de algum fator sofra alterações ao longo do ciclo de vida do projeto.

